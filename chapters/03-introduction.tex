% vim:noet:sts=2:ts=2:sw=2:smarttab:tw=120

\chapter{Introduction}
	\label{chapter:intro}
	Embedded devices are on the rise. While internet-connected refrigerators, which have been journalists' favorite
	prediction for the future, are still a long way off, other turing complete devices have found their way into daily
	life. Taking a look around an ordinary kitchen reveals devices like a fully automatic coffee machine, a digital
	kitchen scale, a dish washer, or even a WiFi-enabled bar code scanner helping users update their shopping list. All of
	those contain microcontrollers, each serving a special purpose. And it seems these are just the first vistas of an
	emerging trend: At the embedded world conference 2014 in Nuremberg, the \emph{Internet of Things}, i.e.\ the
	interconnection of these embedded devices, was a very popular topic. The world might be on the verge of being flooded
	with highly integrated and networked microcontroller-driven devices.

	% See
	% http://www.is2t.com/java-in-embedded-systems/
	% http://www.eetimes.com/document.asp?doc_id=1318980
	% http://www.eetimes.com/document.asp?doc_id=1317511
	% http://www.eetimes.com/document.asp?doc_id=1262568
	% http://www.eetimes.com/document.asp?doc_id=1261676
	% http://www.eetimes.com/document.asp?doc_id=1319569

	These single-purpose embedded devices are commonly programmed in C and the like. However, with increasing software
	complexity of these devices, Java has been emerging as an appropriate alternative for a number of reasons. Java's ease
	of use, large standard library and suitability for complex projects are increasingly called for even in embedded
	software. Its type safe design prevents memory faults that could be caused by out of bounds array accesses or mistakes
	in manual memory management, eliminating a whole class of potential bugs.

	These advantages used to come at the price of reduced performance and increased memory usage due to the need for
	runtime environment and garbage collection, but are challenged by new virtual machines and compilers.
	\emph{Ahead-of-time} compilation, i.e.\ translation to native code before execution, has been gaining popularity
	lately: Google revealed switching its Android platform to a new runtime environment, which is expected to
	\enquote{speed up apps by around 100~\%}~\cite{anthony:13:android-art} using ahead-of-time
	compilation~\cite{lindner:14:android-art}. Microsoft declared a similar intention for its .NET
	platform~\cite{lardinois:14:dotnet-aot}. Oracle and others have developed Java virtual machines targeted at embedded
	systems~\cite{merritt:13:java-for-IoT, maxfield:12:IS2T-JVM}.

	\section{The KESO Multi-JVM}
		\label{sec:intro:keso}
		The KESO Java virtual machine uses ahead-of-time compilation and assumes all application code is available for
		analysis at compile time. This \enquote{closed world} assumption makes aggressive optimizations possible. Built on
		top of an OSEK/VDX~\cite{OSEKSpec223} or AUTOSAR~\cite{autosar:06:sws_os} \gls{rtos}, KESO allows writing
		applications and even device drivers in Java~\cite{thomm:10:jtres}. KESO's compiler \emph{JINO} analyzes the
		application code, tailors the runtime environment (including the Java standard library) to the application's needs
		and emits standards-compliant C to be compiled for the target architecture.

		\Cref{fig:intro:overview} gives an architectural overview of a KESO system. The runtime environment provides an
		abstraction layer on top of the OSEK/VDX or AUTOSAR \gls{rtos} and allows configurable access to specific memory
		addresses for memory-mapped I/O, e.g.\ in device drivers. KESO also provides a mechanism similar to the \gls{jni} to
		execute native code. A KESO system can contain multiple protection realms (so-called \emph{domains}), each of which
		can have a number of tasks, resources, alarms, and \glspl{isr}, its own heap region, and garbage collection
		mechanism. Domains communicate using a \gls{rpc} mechanism called \emph{portals}. The KESO runtime environment
		ensures objects passed through portals cannot interfere with other protection domains.

		\begin{figure}
			\begin{center}
				\documentclass[tikz,convert,crop,boder=100px]{standalone}
% vim:noet:sts=2:ts=2:sw=2:smarttab:tw=120
\usetikzlibrary{arrows,shapes,positioning,calc,decorations.markings,fit}%
%
\pgfdeclarelayer{background}%
\pgfdeclarelayer{foreground}%
\pgfsetlayers{background,main,foreground}%
%
\begin{document}
	\def\radius{0.55em}
	\def\startangle{75}
	\tikzstyle{block} = [
		rectangle,
		draw,
		fill=blue!20,
		text width=5em,
		text centered,
		rounded corners,
		minimum height=2em,
		anchor=north west,
		font=\scriptsize
	]%
	\tikzstyle{microcontroller} = [
		block,
		align = center,
		minimum width = 25em
	]%
	\tikzstyle{oscomponent} = [
		block,
		fill = red!20,
		align = center,
		text width = 10em,
		minimum width = 18em
	]%
	\tikzstyle{kesodomain} = [
		block,
		fill=green!20,
		minimum height=1.5em,
		minimum width=5em
	]%
	\tikzstyle{domainpart} = [
		rectangle,
		draw,
		fill=green!2,
		text width=4em,
		rounded corners,
		minimum height=2em,
		anchor=north west
	]%
	\tikzstyle{gcheapnode} = [
		circle,
		draw,
		inner sep=0.2em
	]%
	\tikzstyle{arr} = [
		draw,
		->,
		shorten >=1pt
	]%
	\tikzstyle{activityarc} = [
		arr,
		->,
		radius=\radius,
		start angle=\startangle,
		delta angle=-330
	]%
	\tikzstyle{connection} = [
		arr,
		thick,
		<->,
		>=latex,
		shorten >=1pt,
		shorten <=1pt
	]%
	\begin{tikzpicture}
		% Microcontroller
		\node (mc) [microcontroller] {Microcontroller};

		% OSEK and KESO Runtime Env
		\node[oscomponent, anchor=south west] at ($(mc.north west)+(0, 0.5em)$) (osek) {OSEK / AUTOSAR};
		\path
			coordinate (kesoNorthWest)       at ($(osek.north west)+(0, 2.5em)$)
			coordinate (kesoNorthMiddleEast) at ($(osek.north east)+(0, 2.5em)$)
			coordinate (kesoNorthEast)       at ($(kesoNorthWest)!(mc.north east)!(kesoNorthMiddleEast)$)
			coordinate (kesoNorth)           at ($(kesoNorthWest)!.5!(kesoNorthEast)$)
			coordinate (kesoSouthWest)       at ($(osek.north west)+(0, 0.5em)$)
			coordinate (kesoSouthEastCorner) at ($(osek.north east)+(.5em, 0.5em)$)
			coordinate (kesoSouth)           at ($(kesoSouthWest)!(kesoNorth)!(kesoSouthEastCorner)$)
			coordinate (kesoSouthEast)       at ($(osek.south west)!(kesoNorthEast)!(osek.south east)$)
			coordinate (kesoSouthMiddleEast) at ($(osek.south east)+(.5em, 0)$)
			;
		\path[oscomponent]
			(kesoNorthWest)
			-- (kesoNorthEast)
			-- (kesoSouthEast)
			-- (kesoSouthMiddleEast)
			-- (kesoSouthEastCorner)
			-- (kesoSouthWest)
			-- cycle;
		\node[anchor = center,
			align = center,
			font = \scriptsize,
			inner sep=.5em] (keso) at ($(kesoNorth)!.5!(kesoSouth)$) {KESO Runtime Environment};
		\coordinate (kesoDeviceBoxNorthWest) at ($(osek.north west)!(kesoSouthMiddleEast)!(osek.north east)$);
		\node[
			anchor = center,
			font = \tiny,
			align = center] at ($(kesoSouthEast)!.5!(kesoDeviceBoxNorthWest)$) {Memory-Mapped I/O\\ Device Drivers};

		% Domains
		\coordinate (domA) at ($(kesoNorthWest)+(0, 9em)$);
		\coordinate (domAWithInnerSep) at ($(domA)+(.5em, -.5em)$);

		% Contents of Domain A
		\begin{pgfonlayer}{foreground}
			% Heap
			% domA is the top left of the domA box, move down to make space for the title, move right by the amount of inner
			% sep of domA + the inner sep of domAHeapWrapper.
			\coordinate (domAHeap) at ($(domA)+(0, -1.7em)+(.9em, 0)$);
			\path (domAHeap)
				+(0.3em, -2.0em) node[gcheapnode, fill=black!20] (gcheapA) {}
				+(1.8em, -2.7em) node[gcheapnode, fill=black] (gcheapB) {}
				+(3.3em, -2.7em) node[gcheapnode, fill=black!20] (gcheapC) {}
				+(1.5em, -1.5em) node[gcheapnode] (gcheapD) {}
				+(2.6em, -1.5em) node[gcheapnode] (gcheapE) {};

			% Connections in the Heap
			\draw[arr] (gcheapB) -- (gcheapA);
			\draw[arr] (gcheapB) -- (gcheapC);
			\draw[arr] (gcheapA) -- (gcheapD);
			\draw[arr] (gcheapC) -- (gcheapE);
		\end{pgfonlayer}
		\node[
			domainpart,
			inner sep=.4em,
			fit = (domAHeap)(gcheapA)(gcheapB)(gcheapC)(gcheapD)(gcheapE)] (domAHeapWrapper) {};
		\begin{pgfonlayer}{foreground}
			\node[
				font = \tiny,
				anchor = north,
				inner sep=.4em] at (domAHeapWrapper.north) (gcheapTitle) {GC/Heap};
		\end{pgfonlayer}

		\begin{pgfonlayer}{foreground}
			% Resources, Alarms, ISRs
			\node[
				domainpart,
				font = \tiny,
				anchor = north west,
				inner sep=.4em,
				align= center] at ($(domAHeapWrapper.north east)+(.2em, 0)$) (domAResources) {\nohyphens{Resources}\\\nohyphens{Alarms}\\ISRs};

			% Tasks
			\coordinate (domATasks) at ($(domAResources.south west)+(0, -.2em)$);
			\coordinate (domATasksWithInnerSep) at ($(domATasks)+(.4em, -.4em)$);
			% Activities
			\foreach \x/\y/\nodename/\nodetext in {.9em/-.8em/domATask1/5, 2.2em/-.8em/domATask2/2, 3.5em/-.8em/domATask3/3} {%
				\draw[->] ($(domATasksWithInnerSep)+(\x, \y)$) arc[activityarc];
				\node[anchor = center, font=\scriptsize] (\nodename) at ($(domATasksWithInnerSep)+(\x,\y)+({-\radius * cos(\startangle)}, {-\radius * sin(\startangle)})$) {\nodetext};
			}
			\coordinate (domATasksForceResourcesWidth) at ($(domAResources.south east)+(0, -.2em)+(-.4em,-.4em)$);
		\end{pgfonlayer}
		\node[domainpart, inner sep=.4em, fit=(domATasksWithInnerSep)(domATask1)(domATask2)(domATask3)(domATasksForceResourcesWidth)] (domATasksWrapper) {};
		\begin{pgfonlayer}{foreground}
			\node[
				font = \tiny,
				anchor = north,
				inner sep=.4em] at (domATasksWrapper.north) (domATasksTitle) {Tasks};
		\end{pgfonlayer}

		\begin{pgfonlayer}{background}
			\node[
				kesodomain,
				anchor = south west,
				fit=(domAWithInnerSep)(domAResources)(domATasksWrapper)(domAHeapWrapper),
				inner sep=.5em] (domAWrapper) {};
		\end{pgfonlayer}
		\begin{pgfonlayer}{foreground}
			% Title
			\node[
				font = \scriptsize,
				anchor = north,
				inner sep = 0.4em] at (domAWrapper.north) (domATitle) {Domain A};
		\end{pgfonlayer}

		\node[kesodomain, anchor = south east] at ($(domAWrapper.south west)!(kesoNorthEast)!(domAWrapper.south east)$) (domC) {Domain C};
		\coordinate (domBHelper) at ($(domAWrapper.north west)!(domC.north west)!(domAWrapper.north east)$);
		\node[kesodomain, anchor = north] at ($(domAWrapper.north east)!.75!(domBHelper)$) (domB) {Domain B};

		%% Connections between Domains
		\draw[connection] (domB) -| node[auto, font=\tiny] {Portal} (domC);
		\draw[connection] (domB) -- node[auto, font=\tiny, swap] {Portal} (domB -| domAWrapper.east);

		%% Connections between Domains and KESO Runtime Env
		\draw[connection] (domAWrapper) -- (kesoNorth -| domAWrapper.south);
		\draw[connection] (domB) -- (kesoNorth -| domB);
		\draw[connection] (domC) -- (kesoNorth -| domC);

		%% Label ``Keso Services''
		\node[anchor = center, font = \scriptsize] at ($(kesoNorth)+(0, 0.9em)$) {KESO Services};
	\end{tikzpicture}%
\end{document}

			\end{center}
			\caption[Schematic overview of a KESO system]{%
				Schematic overview of a KESO system. An OSEK (or AUTOSAR, not depicted) \gls{rtos} runs on a microcontroller. On
				top of the operating system, the KESO runtime environment provides services and abstractions used by the
				application, such as \gls{rpc} primitives or device drivers. Multiple protection realms (\emph{domains}) can
				contain multiple tasks each, have their own resources, heap, and garbage collection method and communicate
				safely using \emph{portals}.}
			\label{fig:intro:overview}
		\end{figure}

		\todonote{A few sentences on why Java is the new node.js of the embedded world. Explain isolation, safety and ease
		of use, put the KESO schematic overview there and explain some of it.}

	\section{Previous Work}
		\label{sec:intro:prev}
		\todonote{Recall what my bachelor thesis was about, find a few similar examples of people that use type-safe
		languages and ahead-of-time compilation to achieve isolation. Outline why escape analysis (or alias analysis in
		general) can not be as precise as it is in type-safe languages in the presence of pointers.}

	\section{Motivation}
		\label{sec:intro:motivation}
		\todonote{Explain the basic idea of scope extension and how it augments EA\@. Also, outline some prospects of how
		scope extension will be all new, shiny and glory.}

	\section{Document Structure}
		\label{sec:intro:document-structure}
		\todonote{Explain what the chapters contain.}
