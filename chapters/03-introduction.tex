% vim:noet:sts=2:ts=2:sw=2:smarttab:tw=120

\chapter{Introduction}
	\label{chapter:intro}
	Embedded devices are on the rise. While refrigerators with internet connections that have been journalists' favorite
	prediction for the future seem to be still a long way off, other turing-complete embedded devices have found their way
	into our daily lives. Take a look around in your kitchen, and you might find a fully automatic coffee machine,
	a digital kitchen scale, or even a WiFi-enabled bar code scanner to help you maintain your shopping list. And it seems
	we're just seeing the first vistas of an emerging trend: The \enquote{Internet of Things} certainly was a popular
	dictum at the embedded world Conference 2014 in Nuremberg. The world might be on the verge of a flood of highly
	integrated embedded devices.

	% See
	% http://www.is2t.com/java-in-embedded-systems/
	% http://www.eetimes.com/document.asp?doc_id=1318980
	% http://www.eetimes.com/document.asp?doc_id=1317511
	% http://www.eetimes.com/document.asp?doc_id=1262568
	% http://www.eetimes.com/document.asp?doc_id=1261676
	% http://www.eetimes.com/document.asp?doc_id=1319569

	Each new device needs software. Traditionally, C and the like are the languages of choice for embedded programming.
	However, Java has been emerging as an option for a number of reasons. First and foremost, the performance degradation
	and large footprint commonly associated with Java are being challenged by new virtual machines and compilers.
	\emph{Ahead-of-time} compilation, i.e.\ translation to native code before execution, as opposed to \emph{just-in-time}
	compilation or byte code interpretation, is becoming increasingly popular: Google revealed its Android platform will
	switch to a new runtime environment \enquote{ART} expected to \enquote{speed up apps by around 100~\%} using
	ahead-of-time compilation compared the previous implementation \enquote{Dalvik}~\cite{lindner:14:android-art,
	anthony:13:android-art}. Microsoft announced a similar initiative for its .NET
	platform~\cite{lardinois:14:dotnet-aot}. Other reasons for Java's rise in the embedded market are the language's ease
	of use, its large standard library and the type-safe design that prevents out-of-bounds accesses or mistakes in manual
	memory management, increasing overall reliability and reducing the number of bugs – properties that are very welcome
	in an embedded environment. Unsurprisingly, Oracle and others offer Java virtual machines for the embedded
	sector~\cite{merritt:13:java-for-IoT, maxfield:12:IS2T-JVM}.

	\todonote{Explain why embedded systems will rule the world in the future. Also explain current trends in embedded
	hardware and how everybody is trying to fit more stuff in less processors. If possible, find some numbers for that (I
	guess Daniel might have some data on that).}

	\section{The KESO Multi-JVM}
		\label{sec:intro:keso}
		\todonote{A few sentences on why Java is the new node.js of the embedded world. Explain isolation, safety and ease
		of use, put the KESO schematic overview there and explain some of it.}

	\section{Previous Work}
		\label{sec:intro:prev}
		\todonote{Recall what my bachelor thesis was about, find a few similar examples of people that use type-safe
		languages and ahead-of-time compilation to achieve isolation. Outline why escape analysis (or alias analysis in
		general) can not be as precise as it is in type-safe languages in the presence of pointers.}

	\section{Motivation}
		\label{sec:intro:motivation}
		\todonote{Explain the basic idea of scope extension and how it augments EA\@. Also, outline some prospects of how
		scope extension will be all new, shiny and glory.}

	\section{Document Structure}
		\label{sec:intro:document-structure}
		\todonote{Explain what the chapters contain.}
