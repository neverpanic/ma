% vim:noet:sts=2:ts=2:sw=2:smarttab:tw=120

\chapter{Escape Analysis}
	\label{chapter:ea}

	\section{Basics}
		\label{sec:ea:basics}
		Summarize how escape analysis works and how the algorithm published by Choi determined whether objects escape
		a method. Explain intraprocedural and interprocedural analysis and give a short example (possibly copy this from the
		bachelor's thesis).

	\section{Improvements}
		\label{sec:ea:improvements}
		Explain the changes I implemented in the escape analysis written in my bachelor's thesis, why they were required or
		beneficial and how they were implemented.

		\subsection{Flow-Sensitivity}
			\label{subsec:ea:improvements:flow-sensitivity}
			Explain what Choi et al.\@ meant by \enquote{we kill only local variables} and how the \texttt{transfer} and
			\texttt{meet} functions are implemented in JINO to achieve flow sensitivity of the escape analysis.

		\subsection{Fixing Incorrect Results}
			\label{subsec:ea:improvements:bug}
			Show that \texttt{Object chooseOne(Object a, Object b)} is not correctly handled by Choi's algorithm if the return
			value of this function is itself returned again. Show what I have done to improve that.

		\subsection{Interprocedural Analysis Optimizations}
			\label{subsec:ea:improvements:opt}
			The work I did in my bachelor's thesis did take rather long especially on large examples. I have implemented a few
			optimizations to make it run faster. These are:

			\subsubsection{No Propagation of Read Operations}
				\label{subsub:ea:improvements:opt:writeonly}
				Ignore read operations and propagate only writes. Explain how that has the potential of cutting down lots of
				nodes generated by vcalls to \texttt{equals(Object other)} (and why that does not work in the implementation we
				use).

			\subsubsection{No Reprocessing of Unchanged Invocations}
				\label{subsub:ea:improvements:opt:quick-fixpoint}
				When iterating to find a fixpoint in a strongly connected component, only re-compute the summary information for
				a method if its callees actually changed; re-use the old results otherwise.

			\subsubsection{Connection Graph Compression}
				\label{subsub:ea:improvements:opt:compression}
				Compress multiple sibling nodes if they likely represent the same node. This gives us a huge speedup without
				giving up too much information (explain how I made sure the relevant information is kept).
