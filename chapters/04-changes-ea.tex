% vim:noet:sts=2:ts=2:sw=2:smarttab:tw=120

\chapter{Escape Analysis}
	\label{chapter:ea}
	KESO's compiler \emph{JINO} uses alias and escape analysis to identify objects whose lifetime is bounded by the
	runtime of their allocating method. The algorithm was implemented in~\cite{lang:12} and is largely based on the work
	of Choi et al.\ in 2003.~\cite{choi:03} The following section contains a brief description of the implementation and
	highlights differences. \Cref{sec:ea:improvements} lists and explains the improvements written for this thesis.

	\section{Basics}
		\label{sec:ea:basics}
		The algorithm starts with alias analysis, which is separated into a method-local (also \emph{intraprocedural})
		analysis and a global (\emph{interprocedural}) analysis. To compute and store alias information, a specialized data
		structure called \gls{cg} is used. For each analyzed method, this graph contains representations of local variables,
		static class members, dynamic instance variables, array indices, and objects. Variables of non-reference type are
		ignored because they do not contribute to alias information.

		\subsection{Intraprocedural Analysis}
			\label{sub:ea:basics:local}
			In intraprocedural analysis, each method in the call graph of an application is traversed and a \gls{cg}
			representation is being computed. It is a key contribution of Choi et al.\ that this representation is independent
			of the calling context. Since the origin of objects might not be known for some objects (e.g.\ if the have been
			passed as argument), a special type of placeholder called \emph{phantom node} is used to represent these objects.
			For pointer analysis as discussed in~\cref{sec:intro:prev} summarizing independent of the aliasing relationships
			in the calling context is impossible~\cite[p.~886]{choi:03}. For each allocation, assignment, field or array
			access, return statement, method invocation, and exception throw, the \gls{cg} is modified appropriately, ensuring
			possible alias relations are represented accurately.

			Nodes in the \gls{cg} have different types: \emph{Object nodes} are added for each encountered allocation site.
			Note that a single object node in the graph might represent multiple objects at runtime because an allocation
			might be executed multiple times (e.g.\ if it is inside a loop). Local variables, static class members, and member
			variables are represented using \emph{local reference nodes}, \emph{global reference nodes}, and \emph{field
			reference nodes} respectively. Array indices are treated like fields and are thus also represented by a field
			reference node. Each reference node can point to a series of object nodes and also to other reference nodes using
			\emph{deferred edges}. Deferred edges are used to simplify updates of the \gls{cg} while processing assignments.
			After intraprocedural analysis, these edges are removed by replacing all incoming deferred edges of a reference
			node with edges to its successors. Different from the work of Choi et al.\ reference nodes with incoming deferred
			but no outgoing edges are preserved without change. \Cref{subsec:ea:improvements:bug} gives the rationale
			underlying this difference. Finally, object nodes can point to field reference nodes, denoting that the pointed
			field exists inside the object where the edge originates.

			Each node in the \gls{cg} has one of three \emph{escape states}, indicating whether a node will outlive its
			allocating method, or even thread. Among these states, a total order exists. \emph{Local} is the lowest state.
			Nodes marked local do not escape the analyzed method. Next after local is \emph{method}. Nodes that outlive
			a method by being returned or assigned to an object passed as parameter are tagged method-escaping. The highest
			escape state is \emph{global} and is given to objects and references that are assigned to static class members or
			thrown as exception. While processing a method's instructions and building the \gls{cg}, operations that cause the
			escape state of one of their parameters to change trigger the appropriate change in the escape states recorded in
			the connection graph. Allocations whose object node representation in the \gls{cg} is tagged local are considered
			for stack allocation.

			See~\cref{lst:ea:basics:ll} for source code corresponding to the \glspl{cg} to be explained in depth. The
			code example is a simple generic linked list. Using common sense we can deduce that, in the absence of a removal
			operation, all list elements will be reachable until the list itself has reached the end of its lifetime.
			Consequently, the only allocation in the given example can not be allocated on the stack, because it must outlive
			the method of its allocation \texttt{addElement}. Due to the structure of the example, intraprocedural analysis
			will not suffice to determine this. Global analysis will be necessary.

			\inputthesiscode{java}{lst:ea:basics:ll}{A simple generic linked list in Java}{%
				A simple generic linked list implementation in Java. Note that this example is more complex than it would have to
				be, especially due to the \texttt{insert} method, for demonstration purposes.\\[.5em]
				An inner class is used to wrap the list entries with references to their successor. The \texttt{addElement} method
				allows insertion of new entries. Internally, \texttt{addElement} uses \texttt{insert}, which enqueues the given
				new element at the start of a list.
			}{examples/LinkedList.java}

			\begin{figure}
				\centering%
				\subcaptionbox{%
					\Acrlong{cg} for \texttt{addElement}. \texttt{insert(list)} and \texttt{insert(elem)} represent the parameters
					passed to the \texttt{insert} method at its invocation. \texttt{<init>(this)} and \texttt{<init>(elem)} do the
					same for the implicit invocation of \texttt{ListElement}'s constructor.%
					\label{subfig:ea:basics:ll:addElement}}[.5\textwidth]{%
						\centering%
						% vim:noet:sts=2:ts=2:sw=2:smarttab:tw=120
\def\hd{6em}%
\def\vd{6ex}%
%
\begin{tikzpicture}[font=\tiny\ttfamily]
	\path
		  (       0,      0)
		 +(-1.8*\hd,   -\vd) node[cgarn,method] (this)       {this}
		 +(-1.8*\hd,      0) node[cgarn]        (insertP0)   {insert(list)}
		 +( -.5*\hd,      0) node[cgon]         (listelem)   {ListElement}
		 +(       0,   -\vd) node[cgarn]        (initP0)     {<init>(this)}
		 +(    -\hd,   -\vd) node[cgarn]        (insertP1)   {insert(elem)}
		 +( 0.8*\hd,   -\vd) node[cgarn,method] (elem)       {elem}
		 +( 0.8*\hd,      0) node[cgarn]        (initP1)     {<init>(elem)}
	;

	\draw[cgde] (insertP0) -- (this);

	\draw[cgpe] ($(initP0.north)+(1em,0)$) |- (listelem);
	\draw[cgpe] ($(insertP1.north)+(-1em,0)$) |- (listelem);

	\draw[cgde] (initP1) -- (elem);
\end{tikzpicture}%
%
				}%
				\hspace{\fill}%
				\subcaptionbox{%
					\Acrlong{cg} for \texttt{insert}. The edge from \texttt{head} to \texttt{obj1} is the one that prevents stack
					allocation of the list element in \texttt{addElement}. Interprocedural analysis needs to propagate this edge
					into the caller context to determine this.\label{subfig:ea:basics:ll:insert}}[.48\textwidth]{%
						\centering%
						% vim:noet:sts=2:ts=2:sw=2:smarttab:tw=120
\def\hd{4.8em}%
\def\vd{6ex}%
%
\begin{tikzpicture}[font=\tiny\ttfamily]
	\path
		  (      0,      0)
		 +(   -\hd,      0) node[cgarn,method] (list)       {list}
		 +(   -\hd,   -\vd) node[cgpon,method] (obj0)       {obj0}
		 +(   -\hd, -2*\vd) node[cgfrn,method] (head)       {head}
		 +(      0,      0) node[cgarn,method] (elem)       {elem}
		 +(      0,   -\vd) node[cgpon,method] (obj1)       {obj1}
		 +(      0, -2*\vd) node[cgfrn,method] (next)       {next}
		 +(    \hd,   -\vd) node[cgpon,method] (oldhead)    {oldhead}
	;

	\draw[cgpe] (list) -- (obj0);
	\draw[cgfe] (elem) -- (obj1);
	\draw[cgfe] (obj0) -- (head);
	\draw[cgpe] (head) -| ($(obj0.east)!.4!(obj1.west)$) -- (obj1);
	\draw[cgfe] (obj1) -- (next);
	\draw[cgpe] (next) -| ($(obj1.east)!.4!(oldhead.west)$) -- (oldhead);
\end{tikzpicture}%
%
				}%
				\caption[The \glsentryshortpl{cg} for \texttt{addElement} and \texttt{insert} after intraprocedural analysis]{%
					The \acrlongpl{cg} for the \texttt{addElement} and \texttt{insert} methods given in~\cref{lst:ea:basics:ll}
					after intraprocedural analysis. Vertices with rounded corners represent \emph{reference nodes}, where
					\emph{field reference nodes} have a red {\color{cgred}\blacksquare}, other reference nodes a blue
					{\color{cgblue}\blacksquare} border. Dotted borders mark artificial reference nodes representing a method's
					parameter or return value. Rectangles with green {\color{cggreen}\blacksquare} borders are \emph{object
					nodes}. If the border is dashed, the node is a \emph{phantom node}. The escape state of nodes is encoded in
					the fill color. White, orange {\color{cgorange}\blacksquare}, and red {\color{cgwarn}\blacksquare} represent
					\emph{local}, \emph{method}, and \emph{global} respectively.}%
				\label{fig:ea:basics:ll}%
			\end{figure}

			See~\cref{fig:ea:basics:ll} for the \acrlongpl{cg} of the methods \texttt{insert} and \texttt{addElement} given
			in~\cref{lst:ea:basics:ll}. The \texttt{addElement} method has two parameters, but only the second one is visible
			in the code listing. Java implicitly passes the \texttt{this} reference as first argument. These parameters are
			represented in \cref{subfig:ea:basics:ll:addElement} by two reference nodes with dotted borders. Since they are
			reachable after the method returns, they are marked as \emph{method}-escaping, denoted by the orange fill color.
			Because the allocation sites of the pointees of both \texttt{this} and \texttt{elem} are unknown, these objects
			are represented using \emph{phantom nodes} (dashed green rectangle). Note that the escape state propagates along
			the edges from \texttt{this} into \texttt{obj0} and from \texttt{elem} into its pointee \texttt{obj1}. The first
			statement in the bytecode representation of \texttt{addElement} is the allocation of a new list element, which
			causes the creation of an object node (green rectangle) in the \gls{cg}. The constructor of the newly created
			\texttt{ListElement} is called with two arguments: a reference to the object and a reference to the given
			parameter \texttt{elem}, represented in the \acrlong{cg} by the \texttt{<init>(this)} and \texttt{<init>(elem)}
			reference nodes. The \texttt{this} parameter of the constructor invocation points to the allocated list element,
			denoted by a solid edge in the graph. The algorithm can deduce that \texttt{<init>(elem)} points to the same
			object as \texttt{elem}, but the pointees of \texttt{elem} are unknown at this point. Adding a deferred edge
			encodes this situation. Finally, processing the invocation of \texttt{insert} creates a similar set of nodes
			\texttt{insert(list)} and \texttt{insert(elem)} pointing to the \texttt{this} reference and the
			\texttt{ListElement} object respectively.

			The graph for \texttt{insert} is given in~\cref{subfig:ea:basics:ll:insert}. The two parameters are again
			represented by a reference node and a phantom node each. The first few statements create the \texttt{next} field
			node below \texttt{obj1} and add a deferred edge to \texttt{head}. This edge is not shown in the graph, because
			the next statement in the code changes the value of \texttt{head}. Depending on whether the analysis is
			flow-sensitive or not, the outcome will differ. The \gls{cg} given uses the flow-sensitive variant, which causes
			all incoming deferred edges of a reference node to be compressed and all outgoing edges to be removed before
			pointing to the new target. Because compression of deferred edges requires at least one edge outgoing from the
			target of the deferred edge, but \texttt{head} did not have any at this point in the analysis, a phantom node
			representing the previous pointees of \texttt{head} was created. It is denoted \texttt{oldhead}
			in~\cref{subfig:ea:basics:ll:insert}. Using flow-insensitive analysis, the deferred edge from \texttt{next} to
			\texttt{head} would have been preserved and later compressed into an edge from \texttt{next} to \texttt{obj1}.
			Finally, the edge from \texttt{head} to \texttt{obj1} is added to represent the effect of the statement in
			line~\ref{line:ea:basics:ll:headassign} in~\cref{lst:ea:basics:ll}, completing the analysis.

			\subsection{Interprocedural Analysis}
				\label{sub:ea:basics:global}

		\todonote{Summarize how escape analysis works and how the algorithm published by Choi determined whether objects
		escape a method. Explain intraprocedural and interprocedural analysis and give a short example (possibly copy this
		from the bachelor's thesis).}

	\section{Improvements}
		\label{sec:ea:improvements}
		\todonote{Explain the changes I implemented in the escape analysis written in my bachelor's thesis, why they were
		required or beneficial and how they were implemented.}

		\subsection{Flow-Sensitivity}
			\label{subsec:ea:improvements:flow-sensitivity}
			\todonote{Explain what Choi et al.\@ meant by \enquote{we kill only local variables} and how the \texttt{transfer}
			and \texttt{meet} functions are implemented in JINO to achieve flow sensitivity of the escape analysis.}

		\subsection{Fixing Incorrect Results}
			\label{subsec:ea:improvements:bug}
			\todonote{Show that \texttt{Object chooseOne(Object a, Object b)} is not correctly handled by Choi's algorithm if
			the return value of this function is itself returned again. Show what I have done to improve that.}

		\subsection{Interprocedural Analysis Optimizations}
			\label{subsec:ea:improvements:opt}
			\todonote{The work I did in my bachelor's thesis did take rather long especially on large examples. I have
			implemented a few optimizations to make it run faster. These are:}

			\subsubsection{No Propagation of Read Operations}
				\label{subsub:ea:improvements:opt:writeonly}
				\todonote{Ignore read operations and propagate only writes. Explain how that has the potential of cutting down
				lots of nodes generated by vcalls to \texttt{equals(Object other)} (and why that does not work in the
				implementation we use).}

			\subsubsection{No Reprocessing of Unchanged Invocations}
				\label{subsub:ea:improvements:opt:quick-fixpoint}
				\todonote{When iterating to find a fixpoint in a strongly connected component, only re-compute the summary
				information for a method if its callees actually changed; re-use the old results otherwise.}

			\subsubsection{Connection Graph Compression}
				\label{subsub:ea:improvements:opt:compression}
				\todonote{Compress multiple sibling nodes if they likely represent the same node. This gives us a huge speedup
				without giving up too much information (explain how I made sure the relevant information is kept).}
