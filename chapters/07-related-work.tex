% vim:noet:sts=2:ts=2:sw=2:smarttab:tw=120

\chapter{Related Work}
	\label{chapter:rel}
	Since the algorithms used in KESO's alias and escape analysis are based on the work of Choi et al.\ published in
	2003~\cite{choi:03:toplas}, behavior, results, and features of the analysis are similar. However, different from their
	work, KESO's compiler avoids resizing a method's stack frame at runtime and offers task-local heaps as an alternative
	optimization backend to stack allocation. For this thesis, a conceptional flaw discussed
	in~\cref{subsec:ea:improve:bug} present in Choi's original work was fixed. Several other performance improvements were
	implemented in~\cref{subsec:ea:improve:opt}. Further applications for escape analysis' results, such as removal of
	unneeded copies in \glspl{rpc} across protection realm boundaries were added to KESO. \Cref{chapter:eea} extended the
	algorithms published in~\cite{choi:03:toplas} with allocation of objects in callers' stack frames.

	\Cref{subsub:ea:improve:opt:compression} presents an alias analysis modification that reduces compile times for large
	specimen considerably by merging sibling nodes. This compression technique is inspired by ideas from Steensgaard's
	almost linear time points-to analysis~\cite{steensgaard:96:popl}. Different from Steensgaards work, KESO's analysis
	does not necessarily compress all sibling nodes pointed to by a common ancestor, but only merges nodes with the same
	escape state to avoid deteriorating the quality of escape analysis results. Object nodes that represent an allocation
	site are not compressed either to preserve the one-to-one mapping between allocation instructions in the intermediate
	code and their corresponding object nodes in the \acrlongpl{cg}.

	Using escape analysis for automatic memory management solves the same problem as region inference. First published by
	Tofte and Talpin in 1994~\cite{tofte:94:popl}, region inference has has seen widespread adaption in later work by
	Henglein~\cite{henglein:01:ppdp}, Grossman~\cite{grossman:02:pldi}, Hallenberg~\cite{hallenberg:02:sigplan},
	Chin~\cite{chin:04:pldi}, and Salagnac~\cite{salagnac:05:aiool, salagnac:07:rtcsa} et al. While the initial
	publication only applied to a call-by-value $\lambda$-calculus, later publications have successfully used similar
	techniques for Standard~ML~\cite{henglein:01:ppdp, hallenberg:02:sigplan}, safe dialects of C~\cite{grossman:02:pldi}
	and Java~\cite{chin:04:pldi, salagnac:05:aiool, salagnac:07:rtcsa}.

	Different from~\cite{grossman:02:pldi, salagnac:07:rtcsa}, KESO's escape analysis is fully automatic and does not
	require source code modifications or developer interaction. Salagnac's work attempts to overcome the problem of region
	size explosion (i.e., region inference placing a lot of objects in the same region that will then not be reclaimed for
	an extensive period of time) by developer interaction and review. While other region based approaches are also
	frequently affected by this problem, KESO's escape analysis-based optimizations do not suffer from it because they do
	not try to avoid garbage collection completely, but rather complement it. If the exact lifetime of an object can not
	be determined, the object is allocated from a garbage-collected heap instead of risking region explosion. In
	fact,~\cite{salagnac:07:rtcsa} gives an example causing region explosion in their approach which would not be
	converted into stack allocations by KESO due to the overlapping liveness region analysis presented
	in~\cref{sub:ea:basics:global}.

	Similar to~\cite{hallenberg:02:sigplan}, the system implemented in this thesis co-exists with garbage collection.
	Hallenberg's approach was implemented for Standard~ML, whereas KESO exclusively uses Java. Both publications by
	Salagnac et al.~\cite{salagnac:05:aiool, salagnac:07:rtcsa} do not use garbage collectors, stating that garbage
	collection is generally unsuitable for real-time Java systems, a view the KESO project does not share.

	Chin's \cite{chin:04:pldi} only supports a subset of Java called \emph{Core-Java} for their analysis, while KESO does
	not impose limitations of the source language features\footnote{even though it does currently not support catching
	exceptions, but this is not a limitation caused by escape analysis, and escape analysis does in fact handle exceptions
	correctly}. Experimental results provided by Chin et al.\ are minimal: The largest example has only 170 lines of
	source code.

	All of~\cite{chin:04:pldi, salagnac:05:aiool, salagnac:07:rtcsa} are written with a context of Java usage in real-time
	systems, which brings them closer to this thesis than other work on region inference. In~\cite{salagnac:05:aiool} the
	goal is avoiding garbage collection entirely, even though the only measurements at runtime still use a garbage
	collector together with region inference. In this benchmark, the actual region-allocated memory is only about 5~\%.
	The approach also only analyzes a subset of the code used by an application, whereas KESO's compiler always analyzes
	the complete application and all library methods used by it. Salagnac's~\cite{salagnac:07:rtcsa} combines
	developer-supported and automatic methods and requires annotations to avoid region size explosion. Furthermore, it
	makes assumptions that have to be verified at runtime~\cite[Sec.~3.1]{salagnac:07:rtcsa}. It is also flow-insensitive
	(even though it operates on code in \gls{ssa} form, which makes this less of an issue) as opposed to KESO's fully
	flow-sensitive approach presented in~\cref{subsec:ea:improve:fs}.
