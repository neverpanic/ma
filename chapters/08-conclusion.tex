% vim:noet:sts=2:ts=2:sw=2:smarttab:tw=120

\chapter{Conclusion}
	\label{chapter:conclusion}
	This thesis strove to improve alias and escape analysis in the KESO Java virtual machine for deeply embedded systems.
	The escape analysis implementation in KESO's compiler \emph{JINO} – initially based on the work of Choi et al.\ in
	2003~\cite{choi:03:toplas} – was improved to be flow-sensitive and run faster. A conceptual flaw that produced
	incorrect analysis results was discovered together with a possible solution in~\cref{subsec:ea:improve:bug}. Using the
	information computed by escape analysis, a number of optimizations such as removal of unneeded copy operations for
	method calls into different protection domains, synchronization optimizations and cycle-aware reference counting were
	discussed or implemented.

	The core part of this thesis concentrates on the application of escape analysis results for automatic memory
	management. Objects whose lifetime is bounded by the runtime of the method they are allocated in are optimized using
	one of two mechanisms. Besides stack allocation, this thesis introduced small task-local heaps with very simple
	automatic memory management as a special case of region-based memory management similar to the \emph{ScopedMemory}
	class in the Real-Time Specification for Java. Furthermore, allocation in a caller's stack frame or task-local heap
	region was explored in~\cref{chapter:eea}.

	Measurements show that in a benchmark suite for real-time Java systems, up to 43.7~\% of allocations are modified to
	automatically manage the allocated objects without garbage collection. At runtime, heap memory usage is more than
	halved in some configurations, a huge improvement from previous versions of this analysis. Additionally, execution is
	sped up by up to 18.7~\% compared to the same configuration without optimizations based on escape analysis.

	The automatic management of a considerable share of objects reduces garbage collector load. Since the lifetimes of
	optimized allocations are organized in a stack manner, external fragmentation can be (and is in KESO's implementation)
	completely avoided. In the context of recent work in KESO concerned with fragmentation-tolerant garbage
	collection~\cite{strotz:14}, reducing the amount of memory that is potentially affected by fragmentation is a welcome
	side effect. Other work concerned with transient errors and software-based mechanisms to detect and correct them
	–~especially in the garbage collector~\cite{taffner:14}~– also benefit from the optimizations in this thesis.

	The alias analysis results in the form of \acrlongpl{cg} have also been used in attempts to improve Java's bad support
	for constant array data in KESO\@. Using the alias graph, constant objects have been identified and placed in read-only
	memory on target architectures that support it~\cite{kuhnle:14}.

	\section*{Future Work}
		\label{sec:conclusion:future-work}
		Despite the good results outlined in~\cref{chapter:eval}, a number of further ideas could still improve results or
		usability of the optimizations. The effectiveness of scope extension is currently limited by virtual method calls,
		which are not modified. Heuristics that use scope extension in its current form when improvements can be expected
		and the additional overhead is low have the potential to further increase the share of objects managed without
		garbage collection. Optimization backends that allow allocation in the memory region corresponding to a calling
		method would lift the requirement of adjusting all call sites and solve the current problem with virtual method
		invocations outlined in~\cref{sub:eea:analysis:virtual}.

		As outlined in~\cref{sec:eea:probs}, scope extension will move allocations from mutually exclusive control flow
		paths to locations that are not mutually exclusive. Since this increases memory usage and potentially slows down
		applications, further analysis to identify these situations and avoid them could be useful. Interference analysis,
		possibly based on Sreedhar's \gls{ssa} based coalescing using $\phi$ congruence classes~\cite{sreedhar:99:sas} could
		identify non-interfering objects – empirical data already revealed this possibility due to a bug in KESO's \gls{ssa}
		deconstruction triggered by non-interfering parameters created by scope extension. A direct method to allocate
		objects in a caller's memory region would be an alternative approach to solving this problem.

		Objects allocated at one of the optimized sites do not support Java's \emph{finalize} methods at the moment. Since
		finalizers could add new references to an object, supporting them would require special analysis support. Additional
		data structures to locate all objects on the stack or in task-local heaps would be required as well as new runtime
		support code that processes objects and calls the \emph{finalize} methods.

		Furthermore, alias analysis could use data flow information associated with references to increase the precision of
		the results without affecting legality. For example, references that are known to be \emph{null} at a given point in
		alias analysis cannot be dereferenced. Any fields read from objects apparently pointed to be these references could
		be ignored – if the reference were actually dereferenced, the program would abort with an exception.

		Finally, the minimum size requirements for stack allocation or task-local heaps could be automatically computed in
		most cases (e.g., in the absence of unbounded recursion or loops). Stack overflows could thus be avoided at compile
		time and programmers would not have to manually determine and configure task-local heap sizes. This idea could be
		implemented in parallel with a \gls{wcet} analysis, because the principles used for both analyses would likely be
		similar in nature.
