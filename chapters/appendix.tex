% vim:noet:sts=2:ts=2:sw=2:smarttab:tw=120

\chapter{Appendix}
	\label{chapter:appendix}

	\section{About the Author}
		\label{sec:appendix:author}
		Clemens Lang was born in August~1988 in Lichtenfels, Germany. He finished his Abitur at \emph{Clavius-Gymnasium
		Bamberg} in May~2007 and interned as web developer at \emph{webDa~Medien~GmbH} from November~2011 to April~2008.
		With an awakened interest in computer science, he worked for \emph{Upjers~GmbH~\&~Co.~KG} before enrolling in
		computer science at the University of Erlangen-Nuremberg in October~2008.

		During his studies, Clemens worked as teaching assistant, instructing undergraduate students in exercises on
		algorithms and data structures in Java, and system programming for Linux in C starting May~2009 until October~2012.
		In summer 2011, he participated in the Google Summer of Code program for the \emph{MacPorts Project}, a package
		manager for open source software targeting OS~X systems. Clemens is an active MacPorts contributor and enthusiast to
		this day.

		After finishing his bachelor's thesis~\cite{lang:12} and earning the Bachelor of Science degree, he took a research
		assistant position at the KESO project of the \emph{System Software Group at FAU}. Clemens' tasks included code
		refactoring, build automation, setting up and administrating continuous integration, gathering and graphing of
		statistics and implementing new optimizations. This thesis concludes the work on alias and escape analysis started
		in his bachelor's thesis.

		During his time at the University of Erlangen-Nuremberg, Clemens held office as one of two student representatives
		in the computer science study commission for five semesters. Furthermore, he contributed to the foundation of
		a fabrication laboratory at the university.

		Current contact details can be found on his website \url{https://neverpanic.de}.

	\clearpage
	\section{Source Code Access}
		\label{sec:appendix:source}
		KESO is distributed under the terms of the \emph{GNU Lesser General Public License}, version 3. The source code is
		published in irregular snapshots available for download from \url{https://www4.cs.fau.de/Research/KESO/#download}.
		The implementation described in this thesis is not available in any snapshots created earlier than 2014-06-23.

		The website also has a documentation section at \url{https://www4.cs.fau.de/Research/KESO/doc/} that can be very
		helpful in starting to work with KESO. The \enquote{First steps} and \enquote{Toolchain} articles are a must-read to
		work with the analyses and optimizations outlined in this thesis.

		\Cref{tbl:appendix:jinoflags} lists the flags for KESO's \emph{JINO} compiler related to this thesis that can be set
		in the \texttt{\$JINOFLAGS} environment variable. Some of the source files that have been written or modified for
		this thesis are given in~\cref{tbl:appendix:src}.

		\begin{table}[p]
			\centering
			{\footnotesize
				\begin{tabular}{p{.35\textwidth}p{.55\textwidth}}
					\textbf{Flag Name}                          & \textbf{Meaning} \\ \hline\hline
					{\scriptsize\texttt{escape\_analysis}} &
						Enables escape analysis and, unless \texttt{tasklocal\_heaps} is set, stack allocation.\\
					{\scriptsize\texttt{production}} &
						Enable production mode, omits debug strings, reduces binary size. Used for measurements.\\
					{\scriptsize\texttt{scope\_extension}} &
						Enable scope extension. See~\cref{chapter:eea}.\\
					{\scriptsize\texttt{stack\_alloc\_stats}} &
						Print statistics about the number of stack allocations and objects' escape states.\\
					{\scriptsize\texttt{superfluous\_portal\_copy\_removal}} &
						Enable removal of unneeded copy operations in portal calls, see~\cref{sub:ea:apps:spca}.\\
					{\scriptsize\texttt{tasklocal\_alloc\_stats}} &
						Print statistics about the number of task-local heap allocations and object's escape states.\\
					{\scriptsize\texttt{tasklocal\_heaps}} &
						Use task-local heaps instead of stack allocation. Requires tasks to have the size of the task local heaps
						configured using the \texttt{LocalHeapSize} property in the KESO configuration file.\\
					{\scriptsize\texttt{tasklocal\_heaps\_avoid\_overlap}} &
						Avoid allocating objects with overlapping liveness regions from task-local heap memory like it is the
						default for stack allocation.
				\end{tabular}
			}
			\caption{JINO configuration flags used by escape analysis and extended escape analysis and their meaning.}
			\label{tbl:appendix:jinoflags}
		\end{table}

		\begin{table}[p]
			\centering

			{\footnotesize
				\begin{tabular}{lp{.40\textwidth}}
					\textbf{File} & \textbf{Contents} \\ \hline\hline
					{\scriptsize\texttt{analysis/EscapeAnalysis.java}} &
						Alias and escape analysis.\\
					{\scriptsize\texttt{analysis/SuperfluousPortalCopyOptimizaton.java}} &
						Removal of unneeded copy operations in portal calls, see~\cref{sub:ea:apps:spca}.\\
					{\scriptsize\texttt{transform/ScopeExtension.java}} &
						Scope extension, see~\cref{chapter:eea}.\\
					{\scriptsize\texttt{transform/StackAllocation.java}} &
						Stack allocation and liveness overlap analysis, see~\cref{sub:ea:basics:global}.\\
					{\scriptsize\texttt{transform/TaskLocalHeapAllocation.java}} &
						Task-local heap allocation, see~\cref{sub:eea:opt:ldh}.\\
				\end{tabular}
			}

			\caption[Paths in the KESO source code that were written or modified for this thesis.]{%
				Paths in the KESO source code that were written or modified for this thesis. All paths are relative to
				\texttt{keso/src/builder/keso/compiler} in the KESO source tree.}
			\label{tbl:appendix:src}
		\end{table}

	% don't use a full chapter, only a section
	\let\chapter\section

	\clearpage
	\bibliographystyle{alpha}
	\bibliography{bibtex/bib,bibtex/all/all}

	\cleardoublepage
	\listoffigures
	\listoftables

	\clearpage
	\tcblistof{mls}{List of Listings}

	% ensure list of algorithms is rendered as section
	\makeatletter
	\section{\listalgorithmcfname}
	\@mkboth{\MakeUppercase\listalgorithmcfname}{\MakeUppercase\listalgorithmcfname}
	\@starttoc{loa}
	\makeatother

	\clearpage
	%\printglossary

	% ensure todonotes are rendered as a section
	\makeatletter
	\if@todonotes@disabled%
	\else
		\section{\@todonotes@todolistname}
		\@starttoc{tdo}
	\fi
	\makeatother
