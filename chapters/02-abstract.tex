% vim:noet:sts=2:ts=2:sw=2:smarttab:tw=120

\cleardoublepage{}

\vspace*{\fill}

\begin{english}[variant = american]
	\begin{abstract}
		Escape analysis can be used for automatic memory management in Java. Based on the work of Choi et al.\ in
		2003~\cite{choi:03:toplas}, this thesis improves the existing escape analysis in KESO, a Java virtual machine for
		deeply embedded systems. Enhancements implemented for this document include flow-sensitive analysis, modifications
		that reduce compile time and a fix for a conceptual flaw in the work of Choi et al\@. Further analysis explored the
		possibility of allocating objects in a callers' stack frames or memory regions.

		Based on the computed analysis results, two different optimization backends are presented and compared: The
		preexisting stack allocation and a new method using task-local heaps automatically managed in a stack-like fashion
		with precise overflow checks. Both methods have predictable allocation and deallocation behavior and can guarantee
		tight upper bounds on their runtime since they do not suffer from external fragmentation. This increases
		predictability, which is beneficial in an embedded real-time Java environment.

		Besides memory management, the thesis discusses other usage possibilities of the escape analysis results, such as
		optimizations in remote procedure calls, synchronization optimizations, and a new theoretical approach to
		cycle-aware reference counting.

		The compiler-assisted memory management pays positively: In a real-time Java benchmark, object lifetime is
		automatically inferred at up to 43.7~\% of all allocation sites. The optimizations reduce heap memory usage, in some
		cases to less than half of what it was without automatic memory management using escape analysis. Additionally, the
		benchmark's time requirements are cut short by up to 18.7~\%.
	\end{abstract}
\end{english}

\vspace*{\fill}
\cleardoublepage{}
\vspace*{\fill}

\begin{german}
	\begin{abstract}
		\todonote{Zusammenfassung schreiben}
	\end{abstract}
\end{german}

\vspace*{\fill}
