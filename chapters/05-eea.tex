% vim:noet:sts=2:ts=2:sw=2:smarttab:tw=120

\chapter{Extended Escape Analysis}
	\label{chapter:eea}
	A standard pattern found in C programs is passing a buffer and its size to a function which will write a computed
	result into the given buffer. Since the location of the buffer is controlled by the calling function, it can be
	allocated in stack memory. In Java, a method would instead allocate a new object on the heap and return a reference to
	it to achieve the same. Using information from alias and escape analysis, objects that escape their method of
	allocation into the caller but no further can be automatically identified. Since the lifetime of these objects can be
	statically determined, the need for garbage collection can be avoided and the memory can be automatically managed
	using compiler-generated code (for example, but not limited to, using stack allocation). This further reduces the load
	of the garbage collection mechanism and can improve worst- and average-case execution times of applications. This
	optimization is called \emph{scope extension} in the following.

	Note that while only stack allocation has been discussed as optimization to manage objects with statically computed
	and bounded lifetimes, it is not the only possibility, and may not be the best. Several other approaches such as
	region-based methods or explicit deallocation operations come to mind. Depending on the nature of the optimization
	used, their unbounded application may lead to problems and can in fact worsen an application's performance.
	Nonetheless, stack allocation will serve as the default backend in the code and the following description of the
	algorithms. Steps in the optimization that are induced by the properties of stack allocations are marked as such.

	In the remainder of this chapter, \cref{sec:eea:idea}~outlines the general idea of the optimization implemented for
	this master's thesis and gives and example showing where, why, and how it can be applied. The
	following~\cref{sec:eea:analysis} discusses in detail which preconditions must be fulfilled, which constellations
	hinder or prevent optimization, and how these shortcomings could be addressed. Two different transformations using the
	analysis results and their advantages and drawbacks are presented in~\cref{sec:eea:opt}, before~\cref{sec:eea:probs}
	concludes this chapter with a consideration of possible problems caused by excessive use of scope extension.

	\section{Algorithmic Idea}
		\label{sec:eea:idea}
		\inputthesiscode{java}{lst:eea:idea:builder}{Example containing a candidate for scope extension}{%
			Example simplified from the CD\textsubscript{j} benchmark from the CD\textsubscript{x} family of
			benchmarks~\cite{kalibera:09:jtres}. The object allocated in \emph{Factory.getBuilder} does not escape
			\emph{Simulation.run}. It can be allocated on the stack of \emph{Simulation.run}.
		}{examples/Factory.java}

		\Cref{lst:eea:idea:builder} shows an example adapted from the source code of the CD\textsubscript{j}
		benchmark~\cite{kalibera:09:jtres} where scope extension can be applied. The \emph{Builder} object allocated in
		\emph{Factory.getBuilder} escapes its allocating method into \emph{Simulation.run}, but is no longer referenced
		after line~\ref{line:eea:idea:builder:lastb}. The runtime of \emph{Simulation.run} is thus an upper bound for the
		lifetime of the object. Consequently, KESO does not have to rely on garbage collection to reclaim the memory used by
		the object, but can instead automatically manage the object's memory using one of the techniques
		from~\cref{sec:eea:opt}.

		All examples discussed so far deal with objects escaping their method of creation via a return operation. Note that
		being returned is not the only way an object can escape: storing references in a field of an object given as
		parameter will also increase the escape state. This case is omitted in all examples for simplicity, but always
		implied.

	\section{Analysis}
		\label{sec:eea:analysis}
		Any object in the \emph{method} escape state partition of a method's \gls{cg} is a candidate for optimization. The
		escape state of the object's representation in the method's callers can be taken into account to decide whether the
		object should be allocated by the caller. Note that since there might be multiple callers and the optimization could
		be applied multiple times (moving allocations up multiple levels in the call hierarchy) considering the escape state
		of the object in the callers' \glspl{cg} is not always a trivial task. For example, the object might escape further
		in some of the callers but not in others. When using stack allocation, even objects that are \emph{local} in
		a calling method might still not be eligible for optimization due to overlapping liveness regions. KESO's stack
		allocation transformation avoids possibly unbounded growth of stack usage by omitting the transformation into
		a stack allocation if multiple objects allocated at the same allocation site are in use simultaneously
		(see~\cref{sub:ea:basics:global} and~\cite[Sec.~3.3]{lang:12} for details).

		When the optimization backend used requires additional parameter passing across invocations, virtual method calls
		need to be handled with special care to avoid breaking their signatures: all candidates for a virtual method
		invocation need to share the same signature before and after optimizing. See~\cref{sub:eea:analysis:virtual} for
		a detailed discussion of virtual method invocations in the context of scope extension.

		Because of the complexity involved in doing so, KESO's implementation does not take the escape state of object
		nodes' equivalents in the callers' \glspl{cg} into account. For each run of the optimization pass, allocations are
		propagated at most a single level up against the direction of the call hierarchy. Therefore, running the pass
		multiple times will increase the maximum scope extension level. Note that it is not necessarily beneficial to run
		the pass often, since it may lead to undesirable results. See~\cref{sec:eea:probs} for a discussion of the
		limitations and problems of scope extension.

		\subsection{Non-virtual Calls}
			\label{sub:eea:analysis:nonvirtual}
			Non-virtual call sites, i.e., those where the invoked method is unique and known at compile time, constitute the
			simple cases of the analysis. Fortunately, the KESO compiler tries to increase the number of non-ambiguous
			invocations by devirtualizing method invocations where a single candidate can be deduced using static
			analysis~\cite[Sec.~3.4]{erhardt:11:jtres}, increasing the number of non-virtual method calls.

			Each object node with a known allocation site (i.e., each non-phantom object node) and an escape state of
			\emph{method} will be optimized in KESO\@. When optimizing allocations of local objects using stack allocation,
			the allocation instruction must be moved into all callers. A reference to the allocated object is instead passed
			to the method on invocation, which uses this reference instead of the reference previously returned by the
			allocation instruction. Each allocation that is optimized using scope extension is copied into all callers and
			executed unconditionally, regardless of whether the allocation sites were in mutually exclusive control flow paths
			before optimization, and hence could never be used at the same time. In some examples, this causes a large number
			of allocations and new method parameters even though only a few are used simultaneously. See~\cref{sec:eea:probs}
			for a detailed analysis of the problem, possible ways to avoid it, and a discussion of the challenges in solving
			it.

		\subsection{Virtual Calls}
			\label{sub:eea:analysis:virtual}
			Virtual method invocations further complicate the decision whether to apply scope extension to an allocation site.
			Since all candidates of a virtual method invocation must share the same signature (i.e., the same parameter and
			return types), a method cannot be optimized individually without considering its siblings when the optimization
			requires adjusting a method's signature (as is the case when using the default stack allocation backend).
			\Cref{fig:eea:analysis:virtual} contains a graphical representation of this problem. Interdependencies between
			methods cause them to form up into groups sharing the same signature. Scope extension, however, depends only on
			the code of the methods in these groups, which is in general unrelated. A single method in such a group could
			cause the modification of its invocations' argument lists, which in turn requires the same changes to all other
			candidates for the modified method calls. Since the modifications are in general unnecessary in all other methods
			than the one causing them, this increases the runtime overhead and possibly allocates memory that is unused in
			most candidates of a virtual invocation.

			\begin{figure}
				\centering%
				% vim:noet:sts=2:ts=2:sw=2:smarttab:tw=120
\def\hd{3em}%
\def\vd{6ex}%
%
\begin{tikzpicture}[font=\tiny\ttfamily, relative]
	\path
		 (     0,     0)
		+(  -\hd,     0) node[callgraphnode]      (a)  {a}
		+(   \hd,     0) node[callgraphnode]      (b)  {b}
		+(-2*\hd,  -\vd) node[callgraphnode]      (n1) {1}
		+(  -\hd,  -\vd) node[callgraphnodefinal] (n2) {2}
		+(     0,  -\vd) node[callgraphnodefinal] (n3) {3}
		+(  +\hd,  -\vd) node[callgraphnodefinal] (n4) {4}
		+( 2*\hd,  -\vd) node[callgraphnode]      (n5) {5}
	;

	\draw[calledge] (a) to (n1);
	\draw[calledge] (a) to (n2);
	\draw[calledge] (a) to (n3);

	\draw[calledge] (b) to (n3);
	\draw[calledge] (b) to (n4);
	\draw[calledge] (b) to (n5);

	\draw[calledge] (a) to (b);
\end{tikzpicture}%
%

				\caption[Call graph showing the complexity of scope extension for virtual method calls]{%
					Call graph showing the complexity of scope extension for virtual method calls. Green
					{\color{cggreen}\blacksquare} vertices mark methods that contain allocations eligible for scope extensions,
					blue {\color{cgblue}\blacksquare} vertices represent other methods. Solid lines are method invocations. Assume
					that both \emph{a} and \emph{b} contain a single virtual method invocation each, i.e., the possible callees
					are \emph{1}–\emph{3} for \emph{a} and \emph{3}–\emph{5} for \emph{b}. Dashed lines point from methods
					eligible for scope extension to methods that must share their signature. Since this relation is transitive,
					nodes \emph{1} through \emph{5} and their invocation sites must be adjusted for each optimization in \emph{2},
					\emph{3}, and \emph{4}.}%
				\label{fig:eea:analysis:virtual}%
			\end{figure}

			Because of the overhead and the complexity inherent to applying this optimization correctly in the presence of
			virtual method calls, KESO does not currently perform scope extension across virtual method calls. Note that some
			of the challenges are caused by properties of the applied optimization. Intermediate code transformations that do
			not require changing a method's signature can simplify the problem. For example, instead of using stack memory,
			a separate thread-local heap section with a simple bump pointer memory management strategy could be used
			(see~\cref{sub:eea:opt:ldh} where this is discussed). Memory could be allocated in the section corresponding to
			a calling method in these thread-local heaps during a method's prologue before creating a new method frame.
			Objects allocated in this region would remain valid until the calling method terminates.

			A different approach to solve the same problem could be not to change the allocations at all (i.e., allocating in
			garbage-collected heap memory) and to modify all callers to explicitly reclaim the objects that are no longer
			used. However, in the presence of a garbage collector this does not necessarily reduce the memory management
			overhead. Marking a section as free reduces neither the time required for the mark phase (the unreferenced section
			will not be marked, whether it was explicitly freed or not) nor the sweep phase (the work done in the sweep step
			is the same, whether it is run by the garbage collector or the explicit statement). As a consequence, the only
			possible improvement achieved by explicit deallocations could be a reduction in the number of garbage collector
			runs. By keeping a list of memory areas that can be re-used immediately, the runtime system could avoid the scan
			phase entirely if enough memory can be reclaimed using the known unused areas.

			Because KESO's current escape analysis summarizes a method's effect independent of any calling contexts, but both
			approaches outlined in the last paragraphs depend on the caller, further analyses would have to be implemented to
			use these ideas.

			In the complete absence of a garbage collector, the scope extension could be used to check manual memory
			management for potential mistakes. In each location where KESO would automatically place a reclaim operation, it
			can check whether the programmer did write the expected instruction. If the object in question is not explicitly
			returned into the memory pool, leakage occurs and the compiler can print a warning. For this application, running
			the scope extension pass multiple times can be beneficial, because due to the lack of modifications, the code size
			does not increase, but the number of objects whose lifetime can be inferred by the compiler grows.

	\section{Optimization}
		\label{sec:eea:opt}
		Based on the results of alias and escape analysis and the decisions presented in~\cref{sec:eea:analysis}, KESO's
		compiler can apply two new optimizing transformations. The first transformation operates on the intermediate code
		representation of the program and moves eligible allocations into a method's callers. To preserve soundness,
		a reference to the allocated object is instead passed as argument at the method's invocation. The newly created
		allocation can potentially use stack memory subsequently. The second new transformation applies in KESO's backend
		where C code is emitted. It is described in~\cref{sub:eea:opt:ldh} and introduces a different concept to avoid
		potential problems with excessive stack usage.

		\subsection{Extending Variable Scope}
			\label{sub:eea:opt:scopeext}
			Extending the scope of variables constitutes the core part of extended escape analysis. The creation of an object
			that will be optimized by the transformation consists of two major instructions on the intermediate code level.
			Since \emph{JINO}'s intermediate code is similar to Java bytecode, these instructions loosely correspond to the
			bytecode instructions generated by the Java compiler for allocations. See~\cref{lst:eea:opt:scopeext:alloc} for
			the sequence of Java bytecode instructions generated for each allocation.

			\inputthesiscode{gas}{lst:eea:opt:scopeext:alloc}{Java bytecode of a complete object allocation}{%
				Java bytecode of a complete object allocation. The \emph{new} instruction allocates new memory. The following
				\emph{dup} operation is required because of the JVM's stack machine semantics. Before \emph{astore\_0} stores
				a reference to the allocated object in the variable number 0, \emph{invokespecial} calls the object's
				constructor and passes a reference to the memory area as first parameter.
			}{examples/Allocation.jbc}

			The first of two instructions allocates a new chunk of memory, initializes any internal data expected by the
			virtual machine (such as runtime type information) and sets the rest of the object's memory to zero to comply with
			Java's semantics. In Java bytecode, this operation is known as \emph{new}. Note that this differs from the
			interpretation of the same keyword at the Java language level, which includes the call to the constructor. If the
			application can be interrupted between stack allocation in the caller and constructor call in the callee (e.g., by
			stop-the-world or on-demand garbage collection or a blocking method call) the referenced memory area must be in
			a defined state. Passing a reference to uninitialized memory (like in C) is not possible unless special
			precautions such as pointer tagging are used.

			The second instruction invokes the object's constructor. The first argument of this call is always a reference to
			the allocated object. Further arguments are passed, if the constructor has any.

			This distinction is important, because the transformation will exclusively deal with the first part. The
			invocation of the constructor is unaffected and will not be moved since that would increase the complexity and
			reduce the number of possible optimization spots. Besides the instruction itself, the invocation's arguments would
			have to be replicated in different methods, which would in turn require copying computations and possibly further
			method calls while preserving Java call semantics.

			Since the invocation of the constructor needs a reference to the allocated object, the allocation instruction is
			replaced with an operation reading a variable. The variable is a newly created method parameter, where the
			parameter type equals the type of the object. Replacing the \emph{new} operation with the instruction reading the
			parameter (one of the \emph{aload} instructions in Java bytecode) is simple but may lead to unnecessary copying.
			However, since \emph{JINO} operates on code in \gls{ssa} form at this point, superfluous variable copies will
			automatically be consolidated in \gls{ssa} deconstruction using Sreedhar's \gls{ssa} based
			coalescing~\cite{sreedhar:99:sas}.

			Adding a new parameter changes the method signature of the callee. This invalidates all existing invocations. As
			a consequence, all callers of an optimized method must be adjusted accordingly. This adjustment consists of
			copying the previously removed allocation instruction right before all invocations and passing the reference
			returned by this operation as new last argument.

			After the pass finishes and all candidates for optimization have been processed, escape analysis is run again.
			This ensures that alias and escape information for the objects allocated at these new allocation sites is up to
			date when it is needed in a subsequent pass optimizing allocations of \emph{local} objects.

		\subsection{Local Task Heaps}
			\label{sub:eea:opt:ldh}
			Turning allocations into stack allocations for automatic memory management is not necessarily the best solution,
			depending on the circumstances. Especially in safety-critical embedded systems, allocating objects and arrays on
			the stack could lead to increased worst-case stack usage estimations. Since the stack space needs to be reserved
			for each task even if it is not going to be used simultaneously, the overall memory requirement can increase
			compared to a system without escape analysis. This situation occurs when the sum of upper bounds is larger than
			the upper bound of the sum. Furthermore, to keep stack usage limited and simplify finding an upper bound of stack
			usage, KESO does not turn allocations whose liveness regions overlap into stack allocations. Overlapping objects
			occur because they are allocated inside a loop and alive after the loop. This requires memory proportional to the
			number of loop runs, where an upper bound might be unknown. To avoid stack overflows, KESO will always serve these
			allocations from heap memory.

			In order to address these shortcomings, an alternative to stack memory is necessary. A special region can be used
			for all objects that can be automatically managed by the compiler. To provide a runtime advantage over the normal
			heap, this region must be exempt from garbage collector sweeps. There should be one logical region for each
			method, while empty regions (i.e., those corresponding to methods without local objects) can be omitted. At the
			end of the method, its associated region can be reclaimed as a whole. To retain the semantics of stack allocations
			and reclaim-on-return, these logical regions should be organized in a stack. One possible implementation of these
			constraints are small specialized heap regions local to each task. Given these local heaps, the logical regions
			are implemented similar to a stack in KESO: Each task-local heap has a fill marker and a maximum fill level. At
			method entry, the fill marker is saved and necessary objects are allocated by moving the fill marker. At method
			exit, the fill marker is reset to its previous value. Saving the fill marker on the stack can be avoided if the
			amount of memory that will be allocated in a function and all alignment offcuts are known at compile time, because
			this knowledge can be used to calculate the value at method entry. KESO's implementation does not currently
			support this optimization. The approach does not require any synchronization for allocations, which constitutes
			another advantage over heap allocation.

			Memory shortages can be detected by checking whether the next operation would move the fill marker above the
			maximum level, preventing unforeseen behavior in case of overflows. Since object allocation on stack no longer
			occurs with this method of region based memory management enabled, finding a tight upper bound for stack usage is
			simplified. With precise and quick checks preventing task-local heap overflows in place, liveness interference
			avoidance can be disabled, further reducing garbage collector load (possibly exceeding amounts proportional to the
			number of affected allocations due to the use in loops). The necessary size of these local heaps can be statically
			configured using results from manual worst-case memory usage analysis. Future work
			(see~\cref{sec:conclusion:future-work}) could automate this process and determine the size of task-local heaps
			automatically.

			It is worth noting that this optimization can only be applied to code that will always be called in a task
			context. Class initializers and methods called from class initializers may not use a task-local heap because they
			are called during startup and not in a specific task. Since tasks and their code are implementations of Java's
			\emph{Runnable} interface in KESO, they have a constructor. While this constructor will also be called from
			outside any task context, task-local allocations can still be enabled by using the task-local heap corresponding
			to the task that is being constructed.

			KESO's implementation uses static analysis to determine which tasks reach a specific method. If only a single task
			uses an allocation site, the dynamic lookup of the task descriptor (which contains the task-local heap descriptor)
			is replaced with a simple address-of operation that can be resolved to a memory address by the compiler and
			linker. If multiple tasks use an allocation, a dynamic lookup is required and preserved. If an allocation is not
			reachable from any task constructor or entry point, it must not use local heaps. KESO's compiler will detect this
			situation and generate an error message to prevent runtime errors. \Glspl{isr} do not use task-local heaps in the
			current implementation. They could use their own task-local heaps or use the task-local heap of the current active
			task they interrupt. The latter solution would, however, increase the worst-case memory usage of all tasks that
			can be interrupted by an \gls{isr}. Portal handler methods use the task-local heap of their caller, which is in
			a suspended state until the portal call finishes.

	\section{Potential Problems}
		\label{sec:eea:probs}
		Applying scope extensions to all candidates does not yield a better program in all cases. A number of situations
		could actually decrease the performance. Heuristics are necessary to avoid these transformations.

		For example, suboptimal results are generated for methods that allocate a large number of objects that are eligible
		for the optimization. A particular specimen exposing this behavior is a generated recursive descent parser used in
		the CD\textsubscript{j} benchmark~\cite{kalibera:09:jtres}: The method that shows the undesirable behavior consists
		of a large distinction of cases where each case allocates and returns an object. Applying scope extension creates
		a new parameter for each object and adds the corresponding allocation to all callers. Besides the overhead caused by
		passing a lot of parameters, this example also exhibits two further problems.

		First, since the control flows in the \emph{switch} statement of the optimized method are mutually exclusive, at
		most a single object is allocated and returned in the example. After scope extension, however, all objects are
		allocated in the caller methods and references are passed for each one, even though only one of the arguments is
		actually used. In this case, the memory usage is thus actually increased by the optimization. This problem could be
		avoided by consolidating memory areas (and the corresponding method parameters) that are used in mutually exclusive
		control flows. Interference analysis is needed to determine this information. Good results can probably be achieved
		using a modification of Sreedhar's $\phi$ congruence classes~\cite{sreedhar:99:sas}, which are already implemented
		in KESO to remove unnecessary copies of variables in \gls{ssa} deconstruction, but are not used in scope extension
		yet. Since consolidated memory areas might be used for objects of different types and sizes, garbage collectors
		would have to support uninitialized chunks of memory as method arguments.

		Summarizing so far, scope extension can increase memory usage due to the allocation of unused objects and it can
		cause subpar performance when a large number of allocations are optimized because of the increased overhead of the
		modified method invocation.

		The necessary modification of a method's call sites induces another set of potential problems. First and foremost,
		optimizing a method with more than one call site will increase code size. Because the allocation instruction is
		removed from the callee and replicated in all callers instead, the optimization is only neutral with respect to the
		code size if a method has exactly one caller.
		\begin{alignat}{3}
			C_\text{after}
				&= C_\text{before} &&+ (r - a)   &&+ c \cdot (a + p) \label{eq:eea:probs:size}\\
			C_\text{after}
				&= C_\text{before} &&+ \Theta(1) &&+ \Theta(c) \label{eq:eea:probs:size-asym}
		\end{alignat}
		\Cref{eq:eea:probs:size} gives a relation between the code sizes before and after applying the optimization. In the
		equation, $r$ denotes the code size of an \emph{aload} (read from variable) operation, $a$ is the size of an
		allocation, $p$ the size of passing an argument to a method and $c$ is the number of callees of the optimized
		method. As~\cref{eq:eea:probs:size-asym} shows, the change in code size is dominated by the number of callers $c$.
		Note that the number of objects allocated at runtime does not change even though the number of allocation
		instructions increases. This is obvious when considering the number of calls to the object's constructor, which is
		not touched by the transformation and hence stays the same.

		The use of appropriate heuristics can prevent the potential problems with methods that have a lot of candidates for
		the optimization. To avoid allocating memory that is not actually used later, interference analysis can be
		implemented. For interference analysis and consolidation, garbage collectors must be adjusted to tolerate chunks of
		uninitialized memory in method arguments. The overhead of passing a lot of parameters can be countered by limiting
		the number of applications of the optimization per method. Code size explosion can be prevented by avoiding the
		optimization for methods whose number of callers is above a certain threshold. Techniques that do not require
		adjusting the calling context (see also~\cref{sub:eea:analysis:virtual}) would completely remove the overhead of
		argument passing and prevent code size from increasing.

		\inputthesiscode{java}{lst:eea:ex}{Scope extension example}{%
			Example for scope extension. The \emph{StringBuilder} object allocated in \emph{buildString} is returned into the
			\emph{run} method. Extending its scope will make it a local object in \emph{run} and subject to optimization.
		}{examples/ScopeExtExample.java}

	\section{Example}
		\label{sec:eea:ex}
		Consider the example given in~\cref{lst:eea:ex} and its Java bytecode representation in~\cref{lst:eea:ex:bc}. The
		\emph{StringBuilder} object allocated in \emph{buildString} escapes its method of creation into \emph{run}.
		Consequently, its \gls{cg} object node representation has a \emph{method} escape state due to its incoming edge from
		the node representing \emph{buildString}'s return value. In the \acrlong{cg} for \emph{run}, the
		\emph{StringBuilder} object is is represented as a phantom node. This phantom node has a \emph{local} escape state
		because it does not escape \emph{run} and is not assigned to any global variables inside \emph{toString}.

		\inputthesiscode{java}{lst:eea:ex:bc}{Bytecode for the scope extension example}{%
			Excerpts from the Java bytecode compiled from the source given in~\cref{lst:eea:ex}, generated using
			\texttt{javac~1.7.0\_55} from OpenJDK~7.
		}{examples/ScopeExtExample.jbc}
		\inputthesiscode{java}{lst:eea:ex:after}{Bytecode after scope extension}{%
			Excerpts from the Java bytecode compiled from the source given in~\cref{lst:eea:ex} after KESO's scope extension.
			The allocation has moved into \emph{run}, where it is \emph{local} and subject to optimizations such as stack
			allocation.
		}{examples/ScopeExtExampleFinished.jbc}

		When scope extension encounters the allocation, it removes the \emph{new} instruction at bytecode position 0 in
		\emph{buildString} and creates an equivalent instruction between bytecode positions 0 and
		1 in \emph{run}. It also adds an additional parameter to the invocation of \emph{buildString}.
		See~\cref{lst:eea:ex:after} for bytecode excerpts from both methods after scope extension. The allocation has been
		moved into \emph{run}, where its representation in the \gls{cg} after another run of alias and escape analysis will
		be \emph{local}. The previous \emph{new} instruction has been replaced with an operation that reads the new
		parameter. \Gls{ssa} deconstruction removes the unnecessary copy that would occur if the \emph{new} operation was
		naively replaced with the appropriate \emph{aload} instruction due to the sequence \emph{aload}, \emph{dup},
		constructor invocation, \emph{astore}.

		The remaining allocation can be turned into a stack allocation or use a task-local heap of the
		\emph{ScopeExtExample} task. Garbage collection is no longer required for the object previously allocated in
		\emph{buildString}. Note that the \emph{toString} method of the \emph{StringBuilder} class contains the allocation
		of a \emph{String} object that escapes its method of creation into \emph{run}. This allocation would also be moved
		into its calling method, despite being part of the standard library. This behavior is not limited to scope
		extension, but is a general feature of KESO\@. All Java library code used in an application is analyzed and tailored
		just like application code.
