% vim:noet:sts=2:ts=2:sw=2:spelllang=en_us
\documentclass[usenames,dvipsnames,smaller]{beamer}

% beamer class setup (i4 style)
\usecolortheme{rose}
\usetheme{i4}

% remove navigation symbols
\beamertemplatenavigationsymbolsempty

% set and load fonts
\usepackage{fontspec}
\setmainfont[
  Ligatures      = TeX,
  ExternalLocation,
  Path           = {../fonts/},
  Extension      = {.otf},
  UprightFont    = {*Regular},
  BoldFont       = {*Bold},
  ItalicFont     = {*Italic},
  BoldItalicFont = {*BoldItalic}]{Charter}
\setsansfont[
  Ligatures      = TeX,
  Scale          = MatchLowercase,
  ExternalLocation,
  Path           = {../fonts/},
  Extension      = {.ttf},
  UprightFont    = {*},
  BoldFont       = {*-Bold},
  ItalicFont     = {*-Oblique},
  BoldItalicFont = {*-BoldOblique}]{Helvetica}
\setmonofont[
  Ligatures = TeX,
  Scale     = MatchLowercase]{Latin Modern Mono}

\usepackage{polyglossia}
\setmainlanguage[latesthyphen=true]{german}

% enable the use of \enquote{} with typographically correct markers
\usepackage[autostyle]{csquotes}

% source code listings
\usepackage{minted}
\usemintedstyle{friendly}

% graphs
\usepackage{standalone}
\usepackage{pgfplots}
\pgfplotsset{compat=1.8}

% load the beamertools package
\usepackage{beamertools}

% load some packages needed for the tikz stuff I'm doing
\usetikzlibrary{positioning}
\usetikzlibrary{arrows}
\usetikzlibrary{shapes}
\usetikzlibrary{calc}

% backup slides shouldn't be numbered. Credits to
% http://tex.stackexchange.com/questions/2541/beamer-frame-numbering-in-appendix
% http://www.stanford.edu/~dgleich/notebook/2009/05/appendix_slides_in_beamer_cont_1.html
\newcommand{\backupbegin}{%
	\newcounter{framenumberappendix}
	\setcounter{framenumberappendix}{\value{framenumber}}
}
\newcommand{\backupend}{%
	\addtocounter{framenumberappendix}{-\value{framenumber}}
	\addtocounter{framenumber}{\value{framenumberappendix}}
}

\definecolor{cgblue}{HTML}{1F78B4}
\definecolor{cggreen}{HTML}{33A02C}
\definecolor{cgred}{HTML}{E31A1C}
\definecolor{cgorange}{HTML}{FDBF6F}
\definecolor{cgwarn}{HTML}{FB9A99}

\tikzstyle{method} = [
	fill=cgorange
]
\tikzstyle{global} = [
	fill=cgwarn
]
\tikzstyle{cgnode} = [
	draw,
	thick
]
\tikzstyle{cgon} = [
	cgnode,
	rectangle,
	draw=cggreen,
	inner xsep=.8em,
	minimum height=3ex
]
\tikzstyle{cgcon} = [
	cgon
]
\tikzstyle{cgpon} = [
	cgon,
	densely dashed
]
\tikzstyle{cgrn} = [
	cgnode,
	rectangle,
	rounded corners,
	inner xsep=.8em,
	minimum height=3ex,
	draw=cgblue
]
\tikzstyle{cglrn} = [
	cgrn
]
\tikzstyle{cggrn} = [
	cgrn,
	global
]
\tikzstyle{cgarn} = [
	cgrn,
	dotted
]
\tikzstyle{cgfrn} = [
	cgrn,
	draw=cgred
]
\tikzstyle{cge} = [
	draw,
	>=latex,
	->,
	thick
]
\tikzstyle{cgde} = [
	cge,
	dashed
]
\tikzstyle{cgpe} = [
	cge
]
\tikzstyle{cgfe} = [
	cge
]
\tikzstyle{callgraphnode} = [
	cgnode,
	anchor=center,
	circle,
	draw=cgblue,
	minimum size=1.5em
]
\tikzstyle{callgraphnodefinal} = [
	callgraphnode,
	draw=cggreen
]
\tikzstyle{calledge} = [
	cgpe
]

\tikzstyle{influence} = [
	calledge,
	dashed
]

\tikzstyle{cycledescriptor}=[
	draw=black,
	rounded corners,
	inner xsep=.8em,
	inner ysep=.5em
]
\title[KESO-EEA]{Compiler-Assisted Memory Management Using Escape Analysis in the KESO JVM}
\subtitle{Vortrag zur Masterarbeit}
\institute{%
	Lehrstuhl für Verteilte Systeme und Betriebssysteme\\
	Friedrich-Alexander-Universität Erlangen-Nürnberg}
\author[cl]{Clemens Lang}
\date[2014-07-11]{11. Juli 2014}

\btset{every block/.style={rounded=false, shadow, text width=.98\textwidth, center}}

\begin{document}
	\begin{frame}[plain]
		\titlepage
	\end{frame}

	\section{Einleitung}
		\begin{frame}{Was ist Fluchtanalyse?}
			\textbf{Was ist Fluchtanalyse?}
			\bi
				\ii Statische Analyse zur Nachverfolgung von Objekten und Referenzen
				\ii Benötigt Alias-Information
				\ii Objekte, deren Lebenszeit durch die der allokierenden Methode beschränkt sind, können durch den Compiler verwaltet werden
			\ei
			\begin{btBlock}<2->{Beispiel}
				\inputminted[fontsize=\footnotesize,linenos,tabsize=2,xleftmargin=1.5em]{java}{Example.java}
			\end{btBlock}
		\end{frame}

		\begin{frame}{Warum Fluchtanalyse?}
			\textbf{Warum Fluchtanalyse?}
			\bi
				\ii Reduzierung der Auslastung des Heaps
					\bi
						\ii[$\Rightarrow$] Reduzierung der GC-Laufzeit, Größe der GC-Datenstrukturen
						\ii[$\Rightarrow$] Verbesserung der Vorhersagbarkeit des Allokationsaufwands
					\ei
				\ii Vermeidung von Fragmentierung
				\ii Verbesserung der Laufzeit
			\ei
		\end{frame}

		\begin{frame}{Fluchtanalyse in KESO}
			\bi
				\ii Basierend auf Choi \emph{et al.}, TOPLAS '03: \enquote{Stack allocation and synchronization optimizations for Java using escape analysis}~\cite{choi:03:toplas}
				\ii Initiale Implementierung in meiner Bachelorarbeit
					\bi
						\ii Gute Ergebnisse, aber lange Übersetzungszeiten und keine Laufzeitverbesserung
						\ii Nur Optimierungen innerhalb einer Methode
					\ei
			\ei
		\end{frame}

		\begin{frame}<beamer>{Überblick}
			\tableofcontents
		\end{frame}

	\AtBeginSection[]{%
		\begin{frame}<beamer>{Überblick}
			\tableofcontents[%
				sectionstyle=show/shaded,
				subsectionstyle=show/show/shaded,
				subsubsectionstyle=show/show/show/shaded
			]
		\end{frame}
	}

	\section{Verbesserungen der existierenden Fluchtanalyse}
		\subsection{Flusssensitivität}
			\begin{frame}{Flusssensitivität}
				\begin{columns}
					\begin{column}{.5\textwidth}
						\inputminted[fontsize=\footnotesize,linenos,tabsize=2,xleftmargin=1.5em]{java}{FlowSensitivity.java}
					\end{column}
					\begin{column}{.5\textwidth}
						% vim:noet:sts=2:ts=2:sw=2:smarttab:tw=120
\def\hd{4.8em}%
\def\vd{6ex}%
%
\begin{tikzpicture}[font=\tiny\ttfamily]
	\path<1-3>
		  (      0,      0)
		 +(   -\hd,      0) node[cglrn] (a)       {a}
		 +(   -\hd,   -\vd) node[cgon]  (objH)    {H}
	;
	\path<2-3>
		  (      0,      0)
		 +(      0,      0) node[cglrn] (c)       {c}
		 +(      0,   -\vd) node[cgon]  (objI)    {I}
	;
	\path<4->
		  (      0,      0)
		 +(    \hd,      0) node[cggrn]        (global)  {FS.global}
		 +(   -\hd,      0) node[cglrn,global] (a)       {a}
		 +(   -\hd,   -\vd) node[cgon,global]  (objH)    {H}
		 +(      0,      0) node[cglrn,global] (c)       {c}
		 +(      0,   -\vd) node[cgon]         (objI)    {I}
	;

	\draw[cgpe] (a) -- (objH);
	\draw<2>[cgpe] (c) -- (objI);
	\draw<3->[cgde] (c) -- (a);
	\draw<4->[cgde] (global) -- (c);
\end{tikzpicture}%

					\end{column}
				\end{columns}
			\end{frame}

		\subsection{Der Doppel-\texttt{return}-Bug}
			\begin{frame}{Der Doppel-\texttt{return}-Bug}
				\bi
					\ii Konzeptioneller Fehler in der Arbeit von Choi et al.
					\ii Bedingt durch einen der Beiträge der Publikation:\\
						Repräsentation des Effekts einer Methode unabhängig vom Aufrufkontext
				\ei
				\begin{columns}
					\begin{column}{.6\textwidth}
						\inputminted[fontsize=\footnotesize,linenos,tabsize=2,xleftmargin=1.5em]{java}{Choose.java}
					\end{column}
					\begin{column}{.4\textwidth}
						\centering
						% vim:noet:sts=2:ts=2:sw=2:smarttab:tw=120
\def\hd{4.8em}%
\def\vd{6ex}%
%
\begin{tikzpicture}[font=\tiny\ttfamily]
	\path<1>
		  (      0,      0)
		 +(   -\hd,      0) node[cgarn]        (param1)  {choose(a)}
		 +(   -\hd,   -\vd) node[cgon]         (obj1)    {O}
		 +(      0,   -\vd) node[cgarn]        (param2)  {choose(b)}
		 +(      0,      0) node[cgon]         (obj2)    {O}
		 +(    \hd,      0) node[cgarn,method] (ret)     {ret}
		 +(    \hd,   -\vd) node[cgpon,method] (phantom) {?}
	;
	\path<2->
		  (      0,      0)
		 +(   -\hd,      0) node[cgarn]        (param1)  {choose(a)}
		 +(   -\hd,   -\vd) node[cgon,method]  (obj1)    {O}
		 +(      0,   -\vd) node[cgarn]        (param2)  {choose(b)}
		 +(      0,      0) node[cgon,method]  (obj2)    {O}
		 +(    \hd,      0) node[cgarn,method] (ret)     {ret}
		 +(    \hd,   -\vd) node[cgpon,method] (phantom) {?}
	;

	\draw[cgpe] (param1) -- (obj1);
	\draw[cgpe] (param2) -- (obj2);
	\draw[cgpe] (ret) -- (phantom);
	\draw<2->[cgpe] (ret) -- (obj1);
	\draw<2->[cgpe] (ret) -- (obj2);

\end{tikzpicture}%

						\vspace*{.2cm}
						\hrule
						\vspace*{.2cm}
						% vim:noet:sts=2:ts=2:sw=2:smarttab:tw=120
\def\hd{4.8em}%
\def\vd{6ex}%
%
\begin{tikzpicture}[font=\tiny\ttfamily]
	\path
		  (      0,      0)
		 +(   -\hd,      0) node[cgarn,method] (param1)  {a}
		 +(   +\hd,      0) node[cgarn,method] (param2)  {b}
		 +(   -\hd,   -\vd) node[cgpon,method] (pobj1)   {?}
		 +(   +\hd,   -\vd) node[cgpon,method] (pobj2)   {?}
		 +(      0,   -\vd) node[cgarn,method] (ret)     {ret}
	;

	\draw[cgpe] (param1) -- (pobj1);
	\draw[cgpe] (param2) -- (pobj2);
	\draw[cgde] (ret) -- (param1);
	\draw[cgde] (ret) -- (param2);

\end{tikzpicture}%

					\end{column}
				\end{columns}
				\bi
					\ii<2-> Behoben durch Identifikation von Referenz-Äquivalenz-Paaren zwischen Aufrufer und Aufgerufenem
				\ei
			\end{frame}

		\subsection{Compilezeitverbesserungen}
			\subsubsection{Ignorieren von Leseoperationen}
				\begin{frame}{Kampf der Compilezeit}{Ignorieren von Leseoperationen}
					\bi
						\ii Lange Übersetzungszeiten nach Double-\texttt{return}-Fix: \textasciitilde 19 min. für 28 kSLOC
							\bi
								\ii[$\Rightarrow$] Schneller analysieren, Präzision beibehalten. \structure{Aber wie?}
							\ei
						\ii<2-> Pathologisches Beispiel: \texttt{equals} in Kombination mit Java Collections API
							\bi
								\ii Gegenseitige rekursive Aufrufe
								\ii Fixpunkt-Iteration in Analyse, Propagation von Knoten in CG der Aufrufer\\
								\structure{$\Rightarrow$} Explosion der Verbindungsgraphen der Methoden die \texttt{equals} nutzen
							\ei
						\ii<3-> Idee: \texttt{equals} liest -- nur durch Schreiboperationen können neue Alias-Beziehungen (und damit flüchtende Objekte) entstehen
							\bi
								\ii Folgerung: In einem CG sind nur die Kanten relevant, die
									\bi
										\ii durch eine Schreiboperation entstehen
										\ii auf einem zyklenfreien Pfad von einem Einsprungpunkt zu einer solchen Kante liegen
									\ei
							\ei
						\ii<4-> Manche Implementierungen von \texttt{equals} enthalten Schreiboperationen! :(
						\ii<4-> Trotzdem deutliche Verbesserungen in Heap-Nutzung und Laufzeit in unseren Benchmarks :)
					\ei
				\end{frame}

			\subsubsection{Connection Graph-Kompression}
				\begin{frame}{Kampf der Compilezeit (II)}{Graph-Komprimierung}
					\bi
						\ii Beobachtungen
							\bi
								\ii Verbindungsgraphen bestehen hauptsächlich aus \emph{phantom nodes}
								\ii Geschwisterknoten repräsentieren häufig das selbe Objekt
							\ei
						\ii Idee: Knoten zusammenfassen, um Graphen zu vereinfachen
						\ii<2-> Adaption von Steensgaards \enquote{Points-to analysis in almost linear time}~\cite{steensgaard:96:popl}
							\bi
								\ii Zwei Knoten werden zusammengeführt, wenn sie eingehende Kanten vom gleichen Knoten haben
								\ii Präzisionsverlust vermeiden: Nur Kompression von \emph{phantom nodes} und innerhalb des gleichen Fluchtstatus
							\ei
						\ii<3-> Übersetzungszeit um eine Größenordnung besser :)
					\ei
				\end{frame}

	\section{Erweiterungen der Fluchtanalyse}
		\subsection{Scope-Erweiterung}
			\begin{frame}{Scope-Erweiterung}
				\textbf{Wie können mehr Objekte durch den Compiler verwaltet werden?}
				\bi
					\ii Aktueller Stand: Automatische Verwaltung methodenlokaler Objekte
					\ii Idee: Erweiterung auf Objekte, die eine Ebene flüchten
				\ei
				\begin{btBlock}<2->{Beispiel: \only<2>{Vorher}\only<3>{Nachher}}
					\begin{overprint}%
						\onslide<2>
						\inputminted[fontsize=\footnotesize,tabsize=2]{java}{ScopeExtExample.java}
						\onslide<3>
						\inputminted[fontsize=\footnotesize,tabsize=2]{java}{ScopeExtExampleFinished.java}
					\end{overprint}
				\end{btBlock}
			\end{frame}

			\begin{frame}{Scope-Erweiterung (II)}
				\textbf{Scope-Erweiterung kurz zusammengefasst}
				\bi
					\ii Kopieren der Allokation in alle Aufrufer
					\ii Übergeben einer Referenz beim Aufruf der Quell-Methode
					\ii Ersetzen der Allokation durch Lesen eines Parameters
					\ii Konstruktoraufruf unmodifiziert
				\ei
				\uncover<2->{%
					\textbf{Probleme und Beschränkungen von Scope-Erweiterung}
					\bi
						\ii Anpassung von Methodensignaturen in virtuellen Aufrufen problematisch
						\ii Allokation von Objekten aus wechselseitig exklusiven Kontrollflüssen
						\ii Wachsende Codegröße bei vielen Aufrufern
						\ii Zusatzaufwand durch Parameterübergabe
					\ei
				}
				\uncover<3->{%
					\textbf{\structure{$\Rightarrow$ Ergebnisse durch Scope-Erweiterung nicht generell besser}}
				}
			\end{frame}

		\subsection{Tasklokale Heaps}
			\begin{frame}{Tasklokale Heaps}
				\bi
					\ii Stack-Allokation nicht notwendigerweise die beste Optimierung
						\bi
							\ii KESO besitzt aktuell keine Stack-Überlauf-Prüfungen
							\ii Stack-Allokation kann zu schlechteren \emph{worst-case}-Abschätzungen führen
							\ii Stack-Rahmen kann nicht zur Laufzeit vergrößert werden ($\Rightarrow$ kein \texttt{alloca(3)})
						\ei
					\ii<2-> Alternative: Separate Region für jeden Task
						\bi
							\ii Explizites Sichern und Wiederherstellen des Kontexts
							\ii Präsize Überlaufprüfungen
							\ii Einfachere Vorhersag- und Messbarbarkeit
							\ii Rahmengröße dynamisch anpassbar
						\ei
				\ei
			\end{frame}

	\section{Evaluation}
		\begin{frame}{Evaluation}{Randbedingungen}
			\bi
				\ii CD\textsubscript{x} Benchmark für Echtzeit-Java~\cite{kalibera:09:jtres}
					\bi
						\ii Statische Daten (Codegröße, Anzahl der optimierten Allokationen)
						\ii Laufzeitmessungen (Zeit, Heap-Nutzung, Instrumentierung der Speicherverwaltung)
					\ei
			\ei
		\end{frame}

		\begin{frame}{Evaluation}{Statische Ergebnisse: Anzahl der Allokationen}
			\bi
				\ii CD\textsubscript{j} \emph{on-the-go}-Variante\\
					\centerline{\documentclass[tikz,convert,crop,boder=100px]{standalone}
% vim:noet:sts=2:ts=2:sw=2:smarttab:tw=120
\usepackage{pgfplots}
\usetikzlibrary{arrows,shapes,positioning,calc}

\begin{document}
	\begin{tikzpicture}[font=\scriptsize]
		\begin{axis}[
			width=.8\textwidth,
			height=3.8cm,
			ybar,
			%bar width=15pt,
			ylabel=\# of allocation sites,
			xlabel=testcase,
			enlarge x limits=0.5,
			ymin=0,
			enlarge y limits={value=0.5,upper},
			symbolic x coords={BT, EA, SE},
			xtick=data,
			nodes near coords,
			nodes near coords align={vertical},
			legend style={draw=none, fill=none, at={(0.5, 0.99)}, anchor={north}, legend columns=-1},
			legend cell align=left,
			label style={font=\scriptsize},
			tick label style={font=\tiny},
			legend style={font=\tiny}
		]
			% all allocations
			\addplot coordinates {(BT, 189) (EA, 146) (SE, 174)};
			\addlegendentry{\;all\;}
			\addplot coordinates {(BT, 65) (EA, 44) (SE, 76)};
			\addlegendentry{\;stack\;}
			\addplot coordinates {(EA, 57) (SE, 89)};
			\addlegendentry{\;task-local\;}
		\end{axis}
	\end{tikzpicture}
\end{document}
}
				\ii CD\textsubscript{j} \emph{simulated multidomain}-Variante\\
					\centerline{\documentclass[tikz,convert,crop,boder=100px]{standalone}
% vim:noet:sts=2:ts=2:sw=2:smarttab:tw=120
\usepackage{pgfplots}
\usetikzlibrary{arrows,shapes,positioning,calc}

\begin{document}
	\begin{tikzpicture}[font=\scriptsize]
		\begin{axis}[
			width=.8\textwidth,
			height=3.8cm,
			ybar,
			%bar width=15pt,
			ylabel=\# of allocation sites,
			xlabel=testcase,
			enlarge x limits=0.5,
			ymin=0,
			enlarge y limits={value=0.5,upper},
			symbolic x coords={BT, EA, SE},
			xtick=data,
			nodes near coords,
			nodes near coords align={vertical},
			legend style={draw=none, fill=none, at={(0.5, 0.99)}, anchor={north}, legend columns=-1},
			legend cell align=left,
			label style={font=\scriptsize},
			tick label style={font=\tiny},
			legend style={font=\tiny}
		]
			% all allocations
			\addplot coordinates {(BT, 851) (EA, 799) (SE, 873)};
			\addlegendentry{\;all\;}
			\addplot coordinates {(BT, 154) (EA, 144) (SE, 172)};
			\addlegendentry{\;stack\;}
			\addplot coordinates {(EA, 162) (SE, 191)};
			\addlegendentry{\;task-local\;}
		\end{axis}
	\end{tikzpicture}
\end{document}
}
			\ei
		\end{frame}

		\begin{frame}{Evaluation}{Statische Ergebnisse: Codegröße}
			\bi
				\ii CD\textsubscript{j} \emph{on-the-go}-Variante\\
					\centerline{\documentclass[tikz,convert,crop,boder=100px]{standalone}
% vim:noet:sts=2:ts=2:sw=2:smarttab:tw=120
\usepackage{pgfplots}
\usetikzlibrary{arrows,shapes,positioning,calc}

\begin{document}
	\begin{tikzpicture}[font=\scriptsize]
		\begin{axis}[
			width=.8\textwidth,
			height=4.2cm,
			xbar,
			xlabel=size in bytes,
			ylabel=testcase,
			enlarge y limits=0.4,
			xmin=0,
			enlarge x limits={0.2,upper},
			symbolic y coords={SE+TLH, SE+stack, EA+TLH, EA+stack, plain},
			ytick=data,
			y tick label style={rotate=45, anchor=east},
			nodes near coords,
			nodes near coords align={horizontal},
			legend style={draw={none}, fill={none}, at={(0.5, 0.99)}, anchor={north}, legend columns=-1},
			legend cell align=left,
			label style={font=\scriptsize},
			tick label style={font=\tiny},
			legend style={font=\tiny}
		]
		  % all allocations
			\addplot coordinates {%
				(46084,SE+TLH)
				(45814,SE+stack)
				(45242,EA+TLH)
				(45138,EA+stack)
				(44342,plain)
			};
			%\addlegendentry{\;on-the-go\;}
		\end{axis}
	\end{tikzpicture}
\end{document}
}
				\ii CD\textsubscript{j} \emph{simulated multidomain}-Variante\\
					\centerline{\documentclass[tikz,convert,crop,boder=100px]{standalone}
% vim:noet:sts=2:ts=2:sw=2:smarttab:tw=120
\usepackage{pgfplots}
\usetikzlibrary{arrows,shapes,positioning,calc}

\begin{document}
	\begin{tikzpicture}[font=\scriptsize]
		\begin{axis}[
			width=.8\textwidth,
			height=4.2cm,
			xbar,
			xlabel=size in bytes,
			ylabel=testcase,
			enlarge y limits=0.4,
			xmin=0,
			enlarge x limits={0.2,upper},
			symbolic y coords={SE+TLH, SE+stack, EA+TLH, EA+stack, plain},
			ytick=data,
			y tick label style={rotate=45, anchor=east},
			nodes near coords,
			nodes near coords align={horizontal},
			legend style={draw={none}, fill={none}, at={(0.5, 0.99)}, anchor={north}, legend columns=-1},
			legend cell align=left,
			label style={font=\scriptsize},
			tick label style={font=\tiny},
			legend style={font=\tiny},
			cycle list shift=1
		]
		  % all allocations
			\addplot coordinates {%
				(245973,SE+TLH)
				(243357,SE+stack)
				(225217,EA+TLH)
				(222569,EA+stack)
				(221393,plain)
			};
			%\addlegendentry{\;simulated\;}
		\end{axis}
	\end{tikzpicture}
\end{document}
}
			\ei
		\end{frame}

		\begin{frame}{Evaluation}{Messungen zur Laufzeit: Heap-Nutzung}
			\centering
			\documentclass[tikz,convert,crop,boder=100px]{standalone}
% vim:noet:sts=2:ts=2:sw=2:smarttab:tw=120
\usepackage{pgfplots}
\usetikzlibrary{arrows,shapes,positioning,calc}

\begin{document}
	\begin{tikzpicture}%[auto, font=\footnotesize]
		\begin{axis}[
			width=\textwidth,
			height=7.8cm,
			ylabel={heap usage in \%},
			xlabel=collision detector iteration,
			xmin=-2,
			xmax=52,
			legend style={draw={none}, at={(0.5, 1.02)}, anchor={south}, legend columns=-1},
			legend cell align=left,
			tick label style={font=\scriptsize},
		]
			% reference line
			\addplot+[domain=0:50,sharp plot,thick, no markers] {1};
			\addlegendentry{\;plain}
			% on the go
			\addplot+[sharp plot] table[x expr={\coordindex + 1}, y index=0, header=false]{../measurements/2014-06-21-onthego-tricore/relmemory-ea_beforethesis.txt};
			\addlegendentry{\;EA+stack BT}
			\addplot+[sharp plot] table[x expr={\coordindex + 1}, y index=0, header=false]{../measurements/2014-06-21-onthego-tricore/relmemory-ea_stack.txt};
			\addlegendentry{\;EA+stack}
			\only<2->{%
				\addplot+[sharp plot] table[x expr={\coordindex + 1}, y index=0, header=false]{../measurements/2014-06-21-onthego-tricore/relmemory-ea_tlh.txt};
				\addlegendentry{\;EA+TLH}
			}
			\only<3->{%
				\addplot+[sharp plot] table[x expr={\coordindex + 1}, y index=0, header=false]{../measurements/2014-06-21-onthego-tricore/relmemory-eea_stack.txt};
				\addlegendentry{\;SE+stack}
			}
		\end{axis}
	\end{tikzpicture}
\end{document}

		\end{frame}

		\begin{frame}{Evaluation}{Messungen zur Laufzeit: Ausführungszeit}
			\centering
			\documentclass[tikz,convert,crop,boder=100px]{standalone}
% vim:noet:sts=2:ts=2:sw=2:smarttab:tw=120
\usepackage{pgfplots}
\usetikzlibrary{arrows,shapes,positioning,calc}

\begin{document}
	\begin{tikzpicture}%[auto, font=\footnotesize]
		\begin{axis}[
			width=\textwidth,
			height=7.8cm,
			ylabel={runtime in \%},
			xlabel=collision detector iteration,
			xmin=-2,
			xmax=52,
			legend style={draw={none}, at={(0.5, 1.02)}, font={\footnotesize}, anchor={south}, legend columns=-1},
			legend cell align=left,
			tick label style={font=\scriptsize},
		]
			% reference line
			\addplot+[domain=0:50,sharp plot,thick, no markers] {1};
			\addlegendentry{\;plain}
			% on the go
			%\addplot+[sharp plot] table[x expr={\coordindex + 1}, y index=0, header=false]{../measurements/2014-06-21-onthego-tricore/reltime-ea_beforethesis.txt};
			%\addlegendentry{\;EA+stack BT}
			\addplot+[sharp plot] table[x expr={\coordindex + 1}, y index=0, header=false]{../measurements/2014-06-21-onthego-tricore/reltime-ea_stack.txt};
			\addlegendentry{\;EA+stack}
			\addplot+[sharp plot] table[x expr={\coordindex + 1}, y index=0, header=false]{../measurements/2014-06-21-onthego-tricore/reltime-ea_tlh.txt};
			\addlegendentry{\;EA+TLH}
			\only<2->{%
				\addplot+[sharp plot] table[x expr={\coordindex + 1}, y index=0, header=false]{../measurements/2014-06-21-onthego-tricore/reltime-eea_stack.txt};
				\addlegendentry{\;SE+stack}
			}
			\only<3->{%
				\addplot+[sharp plot] table[x expr={\coordindex + 1}, y index=0, header=false]{../measurements/2014-06-21-onthego-tricore/reltime-eea_tlh.txt};
				\addlegendentry{\;SE+TLH}
			}
		\end{axis}
	\end{tikzpicture}
\end{document}

		\end{frame}

	\section{Zusammenfassung \& Schluss}
		\begin{frame}{Zusammenfassung}
			\bi
				\ii Fluchanalyse lohnt sich!
					\bi
						\ii CD\textsubscript{j}: Bis zu 43.7~\% der Objekte automatisch verwaltet
						\ii CD\textsubscript{j}: Weniger als 50~\% Heap-Nutzung in gemessenen Abschnitten
						\ii Reduzierung der Fragmentierung (CD\textsubscript{j}: von 21.2~\% auf 9.4~\%)
					\ei
				\ii Scope-Erweiterung nicht generell sinnvoll
					\bi
						\ii Diverse Probleme, z.\,B.\ mit virtuellen Methodenaufrufen
						\ii Verbesserungen durch andere Optimierungen möglich
					\ei
			\ei
		\end{frame}

		\begin{frame}{Ende}
			\begin{center}
				\structure{\LARGE Fragen?}\\[.2em]
				Danke für die Aufmerksamkeit
			\end{center}
		\end{frame}

		\backupbegin
			\begin{frame}{Referenzen}
				\bibliographystyle{alpha}
				\bibliography{../bibtex/bib,../bibtex/all/all}
			\end{frame}
		\backupend
\end{document}
