% vim:noet:sts=2:ts=2:sw=2:smarttab:tw=120

\newcommand{\batitle}{Compiler-Assisted Garbage Collection Using Escape Analysis in the KESO Multi-JVM}
\newcommand{\baauthor}{Clemens Lang}
\newcommand{\basubject}{Informatik}
\newcommand{\babirthdate}{09.\,August~1988}
\newcommand{\babirthlocation}{Lichtenfels}
\newcommand{\baadvisors}{
	Prof. Dr.-Ing. habil. Wolfgang Schröder-Preikschat\\
	Dipl.-Inf. Isabella Stilkerich
}
\newcommand{\bastartdate}{01.\,Januar~2014}
\newcommand{\baenddate}{31.\,Juni~2014}
% language of this document, in a form understood by the latex babel package, see `texdoc babel`, section 1.1
\newcommand{\balanguage}{\selectlanguage{english}}

% layout einstellungen
\NeedsTeXFormat{LaTeX2e}
\documentclass[
	twoside,
	12pt,
	a4paper,
	headsepline,     % linie unter kopfzeilen
%	headings=normal, % größe der überschriften, standard ist viel zu groß
	abstract,        % Überschrift über dem Abstract
	titlepage=false,
	BCOR=1cm,
%	draft
]{scrreprt}

\usepackage[pdftex]{graphicx}
\usepackage{amsmath}
\usepackage{standalone}
\usepackage[usenames,dvipsnames,svgnames,table]{xcolor}
\usepackage{tikz}
\usepackage{pgfplots}
\usepackage{pgfplotstable}
\usepackage{lmodern}
\usepackage{xspace}
\usepackage{makeidx}
\usepackage[T1]{fontenc}
\usepackage[utf8]{inputenc}
\usepackage[ngerman,english]{babel}
\usepackage[automark]{scrpage2}
\usepackage{csquotes}
\usepackage[normalem]{ulem}
\usepackage{multirow}
\usepackage{cite}
\usepackage{caption}
\usepackage{subfig}
\usepackage{paralist}
\usepackage{listings}
\usepackage[shortcuts]{extdash}
\usepackage{todonotes}
\usepackage{tocbibind}
\usepackage[algoruled,linesnumbered,commentsnumbered,vlined]{algorithm2e}
% Hyperref moeglichst als letztes Package laden!
\usepackage{hyperref}
\usepackage[nameinlink]{cleveref}
% glossaries nach hyperref
\usepackage[acronym,acronymlists={main},toc]{glossaries}

% headings
\usepackage{titlesec}
\titleformat{\chapter}[hang]{\fontsize{40}{48}\selectfont\rmfamily}{\thechapter\hspace{15pt}\textcolor{white!75!black}{|}\hspace{15pt}}{0pt}{\fontsize{40}{48}\selectfont\rmfamily}
\titleformat{\section}[hang]{\LARGE\rmfamily}{\thesection\hspace{10pt}\textcolor{white!75!black}{|}\hspace{10pt}}{0pt}{\LARGE\rmfamily}
\titleformat{\subsection}[hang]{\Large\rmfamily}{\thesubsection\hspace{8pt}\textcolor{white!75!black}{|}\hspace{8pt}}{0pt}{\Large\rmfamily}
\titleformat{\subsubsection}[hang]{\large\rmfamily}{\thesubsubsection\hspace{8pt}\textcolor{white!75!black}{|}\hspace{8pt}}{0pt}{\large\rmfamily}
\titleformat{\paragraph}[runin]{\large\bfseries\rmfamily}{\theparagraph\hspace{8pt}\textcolor{white!75!black}{|}\hspace{8pt}}{0pt}{\large\bfseries\rmfamily}
\setkomafont{sectioning}{\rmfamily\bfseries}
\setkomafont{descriptionlabel}{\rmfamily\bfseries}

% tikz libs und settings
\usetikzlibrary{arrows,shapes,decorations,positioning,calc,decorations.markings,pgfplots.groupplots}
\pgfdeclarelayer{background}
\pgfdeclarelayer{foreground}
\pgfsetlayers{background,main,foreground}

% pagestyle
\pagestyle{scrheadings}

% enable glossaries
% compiler-related
\newacronym{llvm}{LLVM}{Low-Level Virtual Machine}
\newacronym{ssa}{SSA}{static single assignment}

% OS and abstraction-related
\newacronym{jni}{JNI}{Java native interface}
\newacronym{rtos}{RTOS}{real-time operating system}
\newacronym{isr}{ISR}{interrupt service routine}
\newacronym{rpc}{RPC}{remote procedure call}
\newacronym{mpu}{MPU}{memory protection unit}
\newacronym{mmu}{MMU}{memory management unit}
\newacronym{mcu}{MCU}{microcontroller unit}
\newacronym{wcet}{WCET}{worst-case execution time}

% CG
\newacronym{cg}{CG}{connection graph}
\newacronym{cgon}{object node}{CG Object Node}
\newacronym{cgcon}{constant object node}{CG Constant Object Node}
\newacronym{cgpon}{phantom object node}{CG Phantom Object Node}
\newacronym{cgrn}{reference node}{CG Reference Node}
\newacronym{cglrn}{local reference node}{CG Local Reference Node}
\newacronym{cggrn}{global reference node}{CG Global Reference Node}
\newacronym{cgarn}{actual reference node}{CG Actual Reference Node}
\newacronym{cgfrn}{field reference node}{CG Field Reference Node}
\newacronym{cgfe}{field edge}{CG Field Edge}
\newacronym{cgpe}{points-to edge}{CG Points-To Edge}
\newacronym{cgde}{deferred edge}{CG Deferred Edge}

\makeglossaries
% disable linking acronyms to their glossary entry
\glsdisablehyper

% use PDFsync
%\synctex=1

% increase emergency stretch to prevent overfull hboxes
\setlength{\emergencystretch}{1em}

% Bilder werden auch im Unterverzeichnis images/ gesucht
\graphicspath{{images/}}

% captions formatieren
\renewcommand{\captionfont}{\small}
\renewcommand{\captionlabelfont}{\small\rmfamily\bfseries}
\setlength{\captionmargin}{1em}
\setlength{\abovecaptionskip}{1em}
\setlength{\belowcaptionskip}{1em}

\setlength{\parskip}{0.2em}

% tiefe der toc
\setcounter{secnumdepth}{3}
\setcounter{tocdepth}{3}

% Listing-Umbegung einrichten
% ACHTUNG: Dieses Paket muss nach den "float", "rotating" und "subfigure" Paketen eingebunden werden!
\lstset{%
	language=Java,                  % choose the language of the code
	basicstyle=\footnotesize,       % the size of the fonts that are used for the code
	numbers=left,                   % where to put the line-numbers
	numberstyle=\footnotesize,      % the size of the fonts that are used for the line-numbers
	stepnumber=1,                   % the step between two line-numbers. If it's 1 each line will be numbered
	numbersep=5pt,                  % how far the line-numbers are from the code
	%backgroundcolor=\color{white}, % choose the background color. You must add \usepackage{color}
	showspaces=false,               % show spaces adding particular underscores
	showstringspaces=false,         % underline spaces within strings
	showtabs=false,                 % show tabs within strings adding particular underscores
	frame=single,                   % adds a frame around the code
	tabsize=2,                      % sets default tabsize to 2 spaces
	captionpos=b,                   % sets the caption-position to bottom
	breaklines=true,                % sets automatic line breaking
	breakatwhitespace=true,         % sets if automatic breaks should only happen at whitespace
	title=\lstname,                 % show the filename of files included with \lstinputlisting; also try caption instead of title
}

\lstdefinelanguage{jino-imcode}{%
	morekeywords={%
		AConstant,%
		Add,%
		Allocation,%
		AMemConstant,%
		AMTConstant,%
		AMTConstantStatic,%
		AObjConstant,%
		AReadArray,%
		AReadLocalVariable,%
		ArgPhi,%
		ArrayClass,%
		ArrayLength,%
		AStoreArray,%
		AStoreLocalVariable,%
		BasicBlock,%
		BasicBlockList,%
		BasicVisitor,%
		BinaryNode,%
		BitAnd,%
		BitOr,%
		BitXor,%
		Branch,%
		BReadArray,%
		BStoreArray,%
		CallerEntry,%
		CallGraphVisitor,%
		Call,%
		CastD2F,%
		CastD2I,%
		CastD2L,%
		CastF2D,%
		CastF2I,%
		CastF2L,%
		CastI2B,%
		CastI2C,%
		CastI2D,%
		CastI2F,%
		CastI2L,%
		CastI2S,%
		Cast,%
		CastL2D,%
		CastL2F,%
		CastL2I,%
		CaughtException,%
		CheckCast,%
		Class,%
		CodeFlattener,%
		CodeWriter,%
		Compare,%
		ConditionalBranch,%
		Constant,%
		CopyVisitor,%
		CReadArray,%
		CStoreArray,%
		DAdd,%
		DataFlowAMem,%
		DataFlowAny,%
		DataFlowArray,%
		DataFlowConstObj,%
		DataFlowConstValue,%
		DataFlowIntRange,%
		DataFlowInvalidObj,%
		DataFlow,%
		DataFlowMessage,%
		DataFlowObj,%
		DataFlowResource,%
		DataFlowSet,%
		DataFlowSHM,%
		DataFlowValidObj,%
		DConstant,%
		DDiv,%
		DeadCodeVisitor,%
		DefUseUpdater,%
		DefVisitor,%
		DescriptorContainer,%
		DGCompare,%
		Div,%
		DLCompare,%
		DMul,%
		DNeg,%
		Domain,%
		DReadArray,%
		DReadLocalVariable,%
		DRem,%
		DStoreArray,%
		DStoreLocalVariable,%
		DSub,%
		EndBlock,%
		Epilog,%
		EQConditionalBranch,%
		EscapeVisitor,%
		ExceptionHandler,%
		FAdd,%
		FConstant,%
		FDiv,%
		FGCompare,%
		Field,%
		FieldRead,%
		FieldWrite,%
		FLCompare,%
		FMul,%
		FNeg,%
		FReadArray,%
		FReadLocalVariable,%
		FRem,%
		FStoreArray,%
		FStoreLocalVariable,%
		FSub,%
		GEConditionalBranch,%
		GetField,%
		GetStatic,%
		GlobalHeap,%
		Goto,%
		GTConditionalBranch,%
		HeapCoffeeBreak,%
		HeapIdleRoundRobin,%
		Heap,%
		HeapRestrictedDomainScope,%
		HeapSlottedGC,%
		IAdd,%
		IBitAnd,%
		IBitOr,%
		IBitXor,%
		IConstant,%
		IDiv,%
		IMul,%
		INeg,%
		InlineVisitor,%
		InstanceOf,%
		InvokeInterface,%
		Invoke,%
		InvokeNonUnique,%
		InvokeSpecial,%
		InvokeStatic,%
		InvokeUnique,%
		InvokeVirtual,%
		IReadArray,%
		IReadLocalVariable,%
		IRem,%
		IShiftLeft,%
		IShiftRight2,%
		IShiftRight,%
		IStoreArray,%
		IStoreLocalVariable,%
		ISub,%
		LAdd,%
		LBitAnd,%
		LBitOr,%
		LBitXor,%
		LCompare,%
		LConstant,%
		LDiv,%
		LEConditionalBranch,%
		ListHead,%
		LMul,%
		LNeg,%
		LocalPortalMethod,%
		LookupSwitch,%
		LReadArray,%
		LReadLocalVariable,%
		LRem,%
		LShiftLeft,%
		LShiftRight2,%
		LShiftRight,%
		LStoreArray,%
		LStoreLocalVariable,%
		LSub,%
		LTConditionalBranch,%
		MethodFrame,%
		Method,%
		MethodNotFound,%
		MethodVisitor,%
		Module,%
		Monitor,%
		Mul,%
		NEConditionalBranch,%
		Neg,%
		NewArray,%
		New,%
		NewMultiArray,%
		NewObjArray,%
		Node,%
		NullCheck,%
		NullConstant,%
		ObjectArrayClass,%
		Operand,%
		Phi,%
		PhiUpdateVisitor,%
		PopReturnAddr,%
		PortalClassInit,%
		PortalClass,%
		PortalConstructor,%
		PortalInit,%
		PortalServiceInit,%
		PortalService,%
		PutField,%
		PutStatic,%
		ReadArray,%
		ReadLocalVariable,%
		Rem,%
		RemotePortalMethod,%
		RenameVisitor,%
		ReplaceVisitor,%
		ReturnSubroutine,%
		ShiftLeft,%
		ShiftRight2,%
		ShiftRight,%
		Slot,%
		SlotPref,%
		SReadArray,%
		SStoreArray,%
		StackOperationDUP2,%
		StackOperationDUP2_X1,%
		StackOperationDUP2_X2,%
		StackOperationDUP,%
		StackOperationDUP_X1,%
		StackOperationDUP_X2,%
		StackOperationPOP2,%
		StackOperationPOP,%
		StackOperationSWAP,%
		StartBlock,%
		StoreArray,%
		StoreLocalVariable,%
		Sub,%
		SystemMethod,%
		TableSwitch,%
		Task,%
		Throw,%
		TypedVisitor,%
		TypedVisitorSimpleAdapter,%
		UnaryNode,%
		UpdateUsesVisitor,%
		Visitor
	},
	moredelim=[il][\emph]{|}
}

% hyperref einrichten
\hypersetup{%
	pdfpagemode=UseNone,%
	pdfpagelayout=TwoPageRight,%
	colorlinks=true,%
	linkcolor=black,%
	citecolor=black,
	urlcolor=black,%
	pdfauthor={\baauthor},%
	pdftitle={\batitle},
	pdfsubject={\batitle},
	pdfkeywords={KESO, embedded systems, Marmelade, Java},
	pdfstartpage=1%
}

% Some useful todonodes-macros
\newcommand{\reword}[1]{%
	\uwave{#1}\todo[color=GreenYellow]{Reword}%
	\xspace%
}
\newcommand{\missingref}[1]{%
	\uwave{$\backslash$ref\{#1\}}%
	\todo[color=Dandelion]{Missing reference}%
	\xspace%
}
\newcommand{\missingcite}[1]{%
	\uwave{[{\tiny\ldots}]}%
	\todo[color=Dandelion]{Missing $\backslash$cite: #1}%
	\xspace%
}
\newcommand{\todonote}[1]{%
	\todo[inline]{#1}%
}
\newcommand{\todobox}[2]{%
	\todo[inline,caption=#1]{%
	\begin{minipage}{\linewidth}%
		#1%
		#2%
	\end{minipage}%
	}%
}
\newcommand{\todomark}[2]{%
	\uwave{#1}\todo{#2}%
	\xspace%
}

% better C++ and C# symbols
\usepackage{relsize}
%c from texinfo.tex
\def\ifmonospace{\ifdim\fontdimen3\font=0pt }
%c C plus plus
\def\C++{%
  \ifmonospace%
    C++%
  \else%
    C\kern-.1667em\raise.30ex\hbox{\smaller{++}}%
  \fi%
  \spacefactor1000 }
%c C sharp
\def\Csharp{%
  \ifmonospace%
    C\#%
  \else%
    C\kern-.1667em\raise.30ex\hbox{\smaller{\#}}%
  \fi%
  \spacefactor1000 }
