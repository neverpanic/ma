% vim:noet:sts=2:ts=2:sw=2:smarttab:tw=120

% layout einstellungen
\NeedsTeXFormat{LaTeX2e}
\documentclass[
	twoside,
	openright,
	cleardoublepage=plain,
	paper=a4,
	headsepline,     % linie unter kopfzeilen
	headings=normal, % größe der überschriften, standard ist viel zu groß
	abstract,        % Überschrift über dem Abstract
	titlepage=false,
	BCOR=1cm
%	draft
]{scrreprt}

% I *want* all lists of stuff to have a number (even the bibliography, despite the authors of KOMA thinking that this
% isn't a good idea). Also, these shouldn't be chapters (again, the KOMA authors tell me they know better…) but sections
% in an Appendix chapter.
\KOMAoption{toc}{listofnumbered}
\KOMAoption{toc}{bibliographynumbered}
\KOMAoption{listof}{leveldown}
\KOMAExecuteOptions{fontsize=12pt}

% Hurenkinder und Schusterjungen verhindern
\clubpenalty10000
\widowpenalty10000
\displaywidowpenalty=10000

\usepackage[pdftex]{graphicx}
% Bilder werden auch im Unterverzeichnis images/ gesucht
\graphicspath{{images/}}

\usepackage{standalone}
\usepackage[usenames,dvipsnames,svgnames,table]{xcolor}

% load tikz and pgfplots
\usepackage{tikz}
\usepackage{pgfplots}
\pgfplotsset{compat=1.8}
\usepackage{pgfplotstable}

% tikz libs and settings
\usetikzlibrary{arrows,shapes,decorations,positioning,calc,decorations.markings,pgfplots.groupplots}
\pgfdeclarelayer{background}
\pgfdeclarelayer{foreground}
\pgfsetlayers{background,main,foreground}

% provide \nohyphens, used in the title page
\usepackage{hyphenat}

% set and load fonts
\usepackage{fontspec}
\setmainfont[
  Ligatures      = TeX,
  ExternalLocation,
  Path           = {./fonts/},
  Extension      = {.otf},
  UprightFont    = {*Regular},
  BoldFont       = {*Bold},
  ItalicFont     = {*Italic},
  BoldItalicFont = {*BoldItalic}]{Charter}
\setsansfont[
  Ligatures      = TeX,
  Scale          = MatchLowercase,
  ExternalLocation,
  Path           = {./fonts/},
  Extension      = {.ttf},
  UprightFont    = {*},
  BoldFont       = {*-Bold},
  ItalicFont     = {*-Oblique},
  BoldItalicFont = {*-BoldOblique}]{Helvetica}
\setmonofont[
  Ligatures = TeX,
  Scale     = MatchLowercase]{Latin Modern Mono}

% load successor package of babel for LuaTeX and load the language files
\usepackage{polyglossia}
\setmainlanguage[variant=american]{english}
\setotherlanguage[latesthyphen=true]{german}

% define a date format for the titlepage
\usepackage[nodayofweek]{datetime}
\newdateformat{citationdate}{\shortmonthname[\THEMONTH] \THEYEAR}

% headings
\usepackage[automark]{scrpage2}
\pagestyle{scrheadings}

% enable the use of \enquote{} with typographically correct markers
\usepackage[autostyle]{csquotes}

% span multiple rows in tables
\usepackage{multirow}

% format caption texts
\usepackage[
	format=hang,
	font={scriptsize},
	labelfont={scriptsize, bf, rm}]{caption}

\usepackage[
	font={scriptsize},
	labelfont={scriptsize, bf, rm}]{subcaption}

% better C++ and C# symbols
\usepackage{relsize}
% adjusted from texinfo.tex
\def\ifmonospace{\ifdim\fontdimen3\font=0pt}
% C++ symbol
\def\C++{%
  \ifmonospace%
    C++%
  \else%
    C\kern-.1667em\raise.20ex\hbox{\smaller{+\kern-.1667em+}}%
  \fi%
  \spacefactor1000}

% allow inline enumerations
\usepackage{paralist}

% nice package to outline algorithms
\usepackage[algoruled,algochapter,titlenumbered,linesnumbered,commentsnumbered,vlined]{algorithm2e}
\SetAlFnt{\scriptsize}

\usepackage{i4coverpage}

% for \blacksquare
\usepackage{amssymb}

% break long URLs
\usepackage[hyphens]{url}

% load this close to the end
\usepackage{hyperref}
\hypersetup{%
	pdfpagemode=UseNone,%
	pdfpagelayout=TwoPageRight,%
	colorlinks=true,%
	linkcolor=black,%
	citecolor=black,
	urlcolor=black,%
	pdfauthor={\thesisauthor},%
	pdftitle={\thesistitle},
	pdfsubject={\thesistitle},
	pdfkeywords={\thesiskeywords},
	pdfstartpage=1%
}

% load glossaries, needs to be after hyperref
\usepackage[acronym,nomain,toc,savewrites,section,numberedsection=autolabel]{glossaries}
% I don't want hyperlinks for now, since I'm not going to print a glossary
\glsdisablehyper
% enable glossaries
% LLVM
\newacronym{llvm}{LLVM}{Low-Level Virtual Machine}

% SSA form
\newacronym{ssa}{SSA}{static single assignment}

% JNI
\newacronym{jni}{JNI}{Java native interface}

% OS, API
\newacronym{rtos}{RTOS}{real-time operating system}

% connection graph
\newacronym{cg}{CG}{connection graph}

% connection graph nodes
\newacronym{cgon}{object node}{CG Object Node}
\newacronym{cgcon}{constant object node}{CG Constant Object Node}
\newacronym{cgpon}{phantom object node}{CG Phantom Object Node}
\newacronym{cgrn}{reference node}{CG Reference Node}
\newacronym{cglrn}{local reference node}{CG Local Reference Node}
\newacronym{cggrn}{global reference node}{CG Global Reference Node}
\newacronym{cgarn}{actual reference node}{CG Actual Reference Node}
\newacronym{cgfrn}{field reference node}{CG Field Reference Node}

% connection graph edges
\newacronym{cgfe}{field edge}{CG Field Edge}
\newacronym{cgpe}{points-to edge}{CG Points-To Edge}
\newacronym{cgde}{deferred edge}{CG Deferred Edge}

\makeglossaries{}

% headings
\usepackage{titlesec}
\titleformat{\chapter}[hang]{\Huge\rmfamily}{%
	\parbox{1.2cm}{\thechapter{}}%
	\parbox{0.8cm}{\centering\textcolor{white!75!black}{|}}}{0pt}{\Huge\rmfamily}
\titleformat{\section}[hang]{\Large\rmfamily}{%
	\parbox{1.2cm}{\thesection{}}%
	\parbox{0.8cm}{\centering\textcolor{white!75!black}{|}}}{0pt}{\Large\rmfamily}
\titleformat{\subsection}[hang]{\large\rmfamily}{%
	\parbox{1.2cm}{\thesubsection{}}%
	\parbox{0.8cm}{\centering\textcolor{white!75!black}{|}}}{0pt}{\large\rmfamily}
\titleformat{\subsubsection}[hang]{\rmfamily}{%
	\parbox{1.2cm}{\thesubsubsection{}}%
	\parbox{0.8cm}{\centering\textcolor{white!75!black}{|}}}{0pt}{\rmfamily}
%\titleformat{\paragraph}[hang]{\rmfamily}{%
%	\parbox{1.2cm}{\theparagraph{}}%
%	\parbox{0.8cm}{\centering\textcolor{white!75!black}{|}}}{0pt}{\rmfamily}
\setkomafont{sectioning}{\rmfamily\bfseries}
\setkomafont{descriptionlabel}{\rmfamily\bfseries}

% increase emergency stretch to prevent overfull hboxes
\setlength{\emergencystretch}{1em}
\setlength{\parskip}{0.2em}

% maximum ToC depth
\setcounter{secnumdepth}{3}
\setcounter{tocdepth}{3}

% Some useful todonotes-macros
\usepackage[colorinlistoftodos]{todonotes}
\usepackage[normalem]{ulem} % for \uwave
\newcommand{\reword}[1]{%
	\parbox{\textwidth}{\uwave{#1}}\todo[color=GreenYellow]{Reword}%
	\xspace%
}
\newcommand{\missingref}[1]{%
	\uwave{$\backslash$ref\{#1\}}%
	\todo[color=Dandelion]{Missing reference}%
	\xspace%
}
\newcommand{\missingcite}[1]{%
	\uwave{[{\tiny\ldots}]}%
	\todo[color=Dandelion]{Missing $\backslash$cite: #1}%
	\xspace%
}
\newcommand{\todonote}[1]{%
	\todo[inline]{#1}%
}
\newcommand{\todobox}[2]{%
	\todo[inline,caption=#1]{%
	\begin{minipage}{\linewidth}%
		#1%
		#2%
	\end{minipage}%
	}%
}
\newcommand{\todomark}[2]{%
	\uwave{#1}\todo{#2}%
	\xspace%
}

% source code listings
\usepackage[chapter]{minted}
\usemintedstyle{friendly}

\usepackage{setspace}
\onehalfspacing

% nice references with automatic naming
\usepackage[nameinlink,capitalize,noabbrev]{cleveref}

% Provide nice boxes for my source code listings. Cats love boxes.
% Note: This needs tcolorbox 3.0, which isn't in texlive 2013
\usepackage{tcolorbox}
\tcbuselibrary{skins}
\tcbuselibrary{minted}
\tcbset{%
	thesiscode/.style 2 args = {%
		listing engine = minted,
		minted style = friendly,
		minted language = #2,
		minted options = {%
			linenos,
			tabsize = 2,
			numbersep = 3mm,
			fontsize = \scriptsize,
			mathescape,
			#1
		},
		colback = black!5!white,
		colframe = black!75!white,
		fonttitle = \bfseries\scriptsize,
		listing and comment,
		enhanced,
		segmentation hidden,
		arc = 0pt,
		left = 5mm,
		floatplacement=t,
		float,
		overlay = {%
			\begin{tcbclipinterior}
				\fill[black!20!white] (frame.south west) rectangle ([xshift=5mm]frame.north west);
			\end{tcbclipinterior}
		}
	},
	thesiscodetitled/.style n args = {4}{%
		thesiscode={#1}{#2},
		label={#3},
		before title={Listing~\thetcbcounter:~},
		title={#4}
	}
}
\newtcblisting[auto counter,number within=chapter,crefname={Listing}{Listings},Crefname={Listing}{Listings}]{thesiscode}[5][]{%
	thesiscodetitled={#1}{#2}{#3}{#4},
	comment={\scriptsize{}#5}}
\newtcbinputlisting[use counter from=thesiscode, list inside=mls]{\inputthesiscode}[6][]{%
	thesiscodetitled={#1}{#2}{#3}{#4},
	comment={\scriptsize{}#5},
	listing file={#6}}
